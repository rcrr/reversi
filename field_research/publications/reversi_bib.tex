\documentclass[twoside, 3p, times, a4paper, 11pt, onecolumn]{elsarticle}

% convert to pdf by running: pdflatex

%%% OTHELLO DIAGRAMS %%%
\usepackage{othelloboard}
\let\Oldothellogrid\othellogrid
\renewcommand{\othellogrid}{\dotmarkings\Oldothellogrid}

% diagram sizes
\newcommand{\scalefactorthreeup}{0.64}
\newcommand{\scalefactortwoup}{0.64}
\newcommand{\scalefactorfourup}{0.48}

% URLs
\usepackage{url}

\usepackage{tikz}
\usetikzlibrary{arrows,decorations.pathmorphing,backgrounds,positioning,fit,petri,calc}

\usepackage{multirow}

\usepackage{amsmath}
\usepackage{graphicx}
\usepackage{float}

\usepackage{epstopdf}

\usepackage[font={small,sf}, labelfont={small,sf,bf}, margin=1cm]{caption}

\usepackage{fancyhdr}
\pagestyle{fancy}
\fancyhead{}
\fancyhead[RO,LE]{\thepage}
\fancyhead[RE,LO]{\slshape \leftmark}
\fancyfoot{}

\begin{document}

\begin{frontmatter}
  \title{AI Software Playing Reversi: a Bibliography Research}

  \author[rcrr]{Roberto Corradini}
  \address[rcrr]{Indipendent researcher, Milan, Italy - rob\_corradini@yahoo.it}

  \begin{abstract}
    In this work I am maintaining the collection of papers, articles, books and more in general publications
    that are of interest in the field of automated agents capable to play the Reversi board game (also known as Othello).
    The development of computer program that raised to the World Championship level and then even further, went on hands in
    hands with the theoretical research in several fields including areas such as game theory, combinatorial
    game searches, logic programming, constraint satisfaction, pattern recognition, parallel computing, and more,
    and of course with the continuous improvements in hardware power.
  \end{abstract}

\end{frontmatter}

\thispagestyle{empty} %removes the footprint preview submitted to ...

\section{Iago}\label{sec:iago}
Computers have always excelled in Othello because average human players cannot envision the drastic board changes caused by moves.
However, few programs played at an advanced level until Paul Rosenbloom created IAGO.

In his article {\it "A World-Championship-Level Othello Program"} \cite{Rosenbloom1982},
published in 1992, Rosenbloom describes the champion Othello program, IAGO. The work described there includes:
(1) a task analysis of Othello;
(2) the implementation of a program based on this analysis and state-of-the-art AI game-playing techniques;
and (3) an evaluation of the program's performance through games played against other programs and comparisons
with expert human play.

The original article can be purchased from Science-Direct at the Internet site \url{www.sciencedirect.com}. An almost
equivalent document can also be found at the Carnegie Mellon University at the URL \url{repository.cmu.edu},
the scanner quality is not as good as the former but it comes at no charge. 


\section{Bill}\label{sec:bill}
The first version of BILL, a program developed by Kai-Fu Lee and Sanjoy Mahajan,
captured first place in the 1985 Waterloo Computer Othello Tournament,
and second place in the 1986 North American Computer Othello Championship.
Moreover, BILL 1.0 consistently defeated IAGO, the program that inspired it.
A full description of BILL 1.0 can be found in {\it "BILL: A table-based knowledge-intensive Othello program"} \cite{Lee1986},
published in 1986 as a technical report by Carnegie Mellon University.

In 1988 the two authors improved BILL and prepared version 2.0 and 3.0 of the program. The main area of improvement has been
the classification of the features used by the evaluation function by mean of Bayesian inference. Their efforts are described
in the article {\it "A pattern classification approach to evaluation function learning"} \cite{Lee1988}, published on Artificial
Intelligence Journal, volume 36 of the same year. The original article can be purchased from Science-Direct at the Internet site
\url{www.sciencedirect.com}. An equivalent pre-print document is made available by the Carnegie Mellon University at the
URL \url{repository.cmu.edu}. 

Finally, a new article summarizing all the achievements done by Lee and Mahajan appeared on Artificial Intelligence in 1990 with
the title {\it "The Development of a World Class Othello Program"} \cite{Lee1990}. It can be obtained again by Science-Direct. 


\section{PAIP}\label{sec:paip}
The book {\it "Paradigms of Artificial Intelligence Programming: Case Studies in Common LISP"} \cite{Norvig1992} written by Peter Norvig
and published by Morgan Kaufmann in 1992, deserves a full chapter on Reversi AI programming.

\bibliographystyle{alpha}
%\bibliographystyle{abbrv}
\bibliography{reversi}
\end{document}
