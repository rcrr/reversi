%
%  two-layer-model.tex
%
%  Copyright (c) 2025, 2026 Roberto Corradini. All rights reserved.
%
%  This file is part of the reversi program
%  http://github.com/rcrr/reversi
%
%  This program is free software; you can redistribute it and/or modify it
%  under the terms of the GNU General Public License as published by the
%  Free Software Foundation; either version 3, or (at your option) any
%  later version.
%
%  This program is distributed in the hope that it will be useful,
%  but WITHOUT ANY WARRANTY; without even the implied warranty of
%  MERCHANTABILITY or FITNESS FOR A PARTICULAR PURPOSE. See the
%  GNU General Public License for more details.
%
%  You should have received a copy of the GNU General Public License
%  along with this program; if not, write to the Free Software
%  Foundation, Inc., 59 Temple Place - Suite 330, Boston, MA  02111-1307, USA
%  or visit the site <http://www.gnu.org/licenses/>.
%

%
% The most part of the Reversi source code and documentation is written in English.
% This file started with the specific intention to experiment new evolutions for the
% program and for the design of the Machine-Learning Model dealing with the help and the
% interaction with AI agents.
% It has become evident that the language of choice is not that relevant for the AI, and
% it is so easy to get a quality translation from AI today, that you are not minding me
% to choose Italian here for the seek of my confort instead of English.
%

%
% TO-DO
% - Completare la definizione dei Pattern.
% - Documentare tutta la evaluation_function.
% - Codice per la creazione del modello e per il calcolo delle prestazioni.
% - Definizione del nuovo modello RNNM (Reversi Neural Network Model)
%


\documentclass{article}
\renewcommand{\familydefault}{\sfdefault}
\usepackage{sfmath}

\usepackage{amsmath}
\usepackage{othelloboard}
\usepackage{subcaption}
\usepackage{graphicx}

\usepackage[a4paper, left=2cm, right=2cm, top=3cm, bottom=3cm]{geometry}

\usepackage{listings}
\usepackage[dvipsnames]{xcolor}
\definecolor{mauve}{rgb}{0.88, 0.69, 1.0}
\usepackage{fancyvrb}
\usepackage{enumitem}

\usepackage{textcomp}
\usepackage[T1]{fontenc}
\usepackage{url}

\usepackage{bm}
\usepackage[utf8]{inputenc}

\usepackage{xcolor}
\usepackage{hyperref}

\hypersetup{
    colorlinks=true,      % Rimuove i riquadri e colora il testo
    linkcolor=blue,       % Colore per i riferimenti interni (sezioni, figure, equazioni)
    citecolor=blue,       % Colore per le citazioni bibliografiche
    filecolor=magenta,    
    urlcolor=cyan,
    pdftitle={Pattern Symmetries in Reversi},
}

\usepackage{amssymb}

\lstset{
  language=SQL,
  basicstyle=\ttfamily\tiny,
  keywordstyle=\color{blue},
  stringstyle=\color{mauve},
  commentstyle=\color{gray},
  numbers=none,
  breaklines=true,
  showstringspaces=false,
  frame=single
}

\lstset{
  language=bash,
  basicstyle=\ttfamily\tiny,
  keywordstyle=\color{blue},
  commentstyle=\color{gray},
  stringstyle=\color{mauve},
  breaklines=true,
  frame=single,
  showstringspaces=false
}

\lstset{
  language=C,
  basicstyle=\ttfamily\tiny,
  keywordstyle=\color{blue},
  stringstyle=\color{mauve},
  commentstyle=\color{gray},
  numbers=none,
  breaklines=true,
  showstringspaces=false,
  frame=single,
  literate={<<}{{$\ll$}}2 {>>}{{$\gg$}}2
}

\usepackage{datatool}
% Carichiamo il database dei dati dei pattern
\DTLloaddb{master_data_patterns}{mdp.csv}

% Commentare la riga per avere il PDF completo.
\includeonly{tlm_intro,tlm_appendix_catalogo_dei_pattern}

% 
% M-:
% (setq preview-scale-function (lambda () 1.5))
%

%
% Per scrivere le vocali accentate in Italiano:
%
% $ setxkbmap -layout us -variant intl
%

\title{Reversi}
\author{Roberto Corradini \\[0.2cm]
\small \texttt{rob\_corradini@yahoo.com} \\[0.1cm]
\url{https://github.com/rcrr/reversi}}
\date{\today}

\begin{document}
\maketitle
\vspace{1cm}
\begin{abstract}
\normalsize
Reversi è un ecosistema di programmi e strumenti informatici per studiare il gioco da tavolo Othello, anche chiamato Reversi.
L'intento è studiare, capire, sviluppare algoritmi e modelli che possano progredire nella realizzazione di un agente
sintetico ed autonomo che sia in grado di giocare una partita in maniera autorevole. E in prospettiva di tendere alla
risoluzione matematica del gioco, o quantomeno ad una sua approssimazione probabilistica.
I programmi sono scritti principalmente in ANSI C, Python, SQL, con altri linguaggi come Common Lisp, Java, Clojure, R usati
per sperimentazioni, ma non rilevanti nella sostanza.
Il corpo principale del programma è sviluppato in C, e comprende l'endgame solver, il generatore di partite random, il
solutore integrato con il database REGAB, e le funzioni di risoluzione del modello di ML basato sulla Regressione Logistica.
Il Database SQL REGAB, sviluppato in PostgreSQL, ed in linguaggio procedurale PL/pgSQL, contiene le posizioni random
che poi vengono risolte e taggate con il valore vero del game.
Il codice in C è stato poi portato in Python, per poter operare più agilmente con l'ecosistema attivo del Machine-Learning, AI,
calcolo scientifico, acceleratori GPU, e tutto quanto serve per la ricerca di frontiera.
Il codice in C rimane più performante per la risoluzione delle posizioni, il codice in Python è più consono allo sviluppo
dei Modelli per la definizione di evaluation function più evolute.
\end{abstract}

%
%  tlm_intro.tex
%
%  Copyright (c) 2026 Roberto Corradini. All rights reserved.
%
%  This file is part of the reversi program
%  http://github.com/rcrr/reversi
%
%  This program is free software; you can redistribute it and/or modify it
%  under the terms of the GNU General Public License as published by the
%  Free Software Foundation; either version 3, or (at your option) any
%  later version.
%
%  This program is distributed in the hope that it will be useful,
%  but WITHOUT ANY WARRANTY; without even the implied warranty of
%  MERCHANTABILITY or FITNESS FOR A PARTICULAR PURPOSE. See the
%  GNU General Public License for more details.
%
%  You should have received a copy of the GNU General Public License
%  along with this program; if not, write to the Free Software
%  Foundation, Inc., 59 Temple Place - Suite 330, Boston, MA  02111-1307, USA
%  or visit the site <http://www.gnu.org/licenses/>.
%


\section{Introduzione e Ambiente di Sviluppo}
Questo documento è la guida per descrivere le basi del modello di Machine-Learning adottato dal
programma Reversi. Il programma è vasto ed articolato, quì si cerca di sintetizzare al meglio le informazioni
basilari per ragionare sullo sviluppo di un modello più performante della valutazione del risultato del gioco,
più sinteticamente chiamata ``evaluation\_function'' o ``ef''.
\begin{itemize}
\item Per cominciare a lavorare avviare \texttt{EMACS} e lanciare una shell con il comando \texttt{``M-x shell''}, quindi posizionarsi
  nella cartella \texttt{\$REVERSI\_HOME/c} e rigenerare il file PDF con il comando \texttt{``make latex-ai''}.
  Per proseguire nella documentazione del modello aprire ed editare il presente documento \texttt{\$REVERSI\_HOME/c/ai/two-model-layer.tex}.
\item Sempre su \texttt{EMACS} aprire una seconda shell con il comando \texttt{``C-u M-x shell''}e posizionarsi nella cartella \texttt{\$REVERSI\_HOME/c}.
  Qui attivare l'ambiente python con il comando \texttt{``source py/.reversi\_venv/bin/activate''} e di seguito
  lanciare gli unit test con il comando \texttt{``PYTHONPATH="./py" python3 -m unittest test.test\_domain''}.
\item Aprire il file \texttt{\$REVERSI\_HOME/c/py/twoml/domain.py} per lavorare sulla logica applicativa, e
  il file\linebreak \texttt{\$REVERSI\_HOME/c/py/test/test\_domain.py} per aggiornare gli unit test.
\item Aprire quindi una terza riga di comando, sempre in \texttt{EMACS} ma di tipo \texttt{term}, con il comando \texttt{``C-u M-x ansi-term''}, e quì eseguire il comado
  \texttt{``gemini''} per avviare il collaboratore sintetico, che in questo contesto chiameremo \texttt{Nia}, diminutivo di Urania, la musa dell'astronomia e della geometria.
  Con il comando \texttt{``@ai/two-model-layer.tex''} fornire a Nia il contesto per poi proseguire nei ragionamenti e nello sviluppo
  del Modello e del Software che lo implementa.
\end{itemize}

%
%  tlm_rappresentazione_del_dominio.tex
%
%  Copyright (c) 2026 Roberto Corradini. All rights reserved.
%
%  This file is part of the reversi program
%  http://github.com/rcrr/reversi
%
%  This program is free software; you can redistribute it and/or modify it
%  under the terms of the GNU General Public License as published by the
%  Free Software Foundation; either version 3, or (at your option) any
%  later version.
%
%  This program is distributed in the hope that it will be useful,
%  but WITHOUT ANY WARRANTY; without even the implied warranty of
%  MERCHANTABILITY or FITNESS FOR A PARTICULAR PURPOSE. See the
%  GNU General Public License for more details.
%
%  You should have received a copy of the GNU General Public License
%  along with this program; if not, write to the Free Software
%  Foundation, Inc., 59 Temple Place - Suite 330, Boston, MA  02111-1307, USA
%  or visit the site <http://www.gnu.org/licenses/>.
%


\section{Rappresentazione del Dominio}

\subsection{La scacchiera (Board)}

La BOARD Reversi è composta da 64 celle, otto righe ed otto colonne, numerate da 0, in alto a sinistra (North-West, NW),
a 63, in basso a destra (South-East, SE).

La configurazione della BOARD è rappresentata con due interi a 64 bit (indifferentemente senza segno \texttt{uint64} o con segno \texttt{int64}).
Ognuno dei 64 bit è associato ad una cella. Il bit meno significativo è associato ad \texttt{a1}, quello più significativo ad \texttt{h8},
procedendo da sinistra verso destra per righe.

Il primo intero rappresenta le celle del \texttt{GIOCATORE (PLAYER)} che deve effetuare la mossa, il secondo le celle dell'\texttt{AVVERSARIO (OPPONENT)}.

Con due valori \texttt{int64} viene quindi rappresentata la \texttt{POSIZIONE (GAME\_POSITION)}, come mostrato in figura~\ref{fig:reversi-bitboard-patterns}.

Unico vincolo, o invariante, della rappresentazione è la formula: \texttt{PLAYER \&\& OPPONENT == 0},
che formalizza il divieto di avere sia un disco del giocatore che uno dell'avversario contemporaneamente nella stessa cella.

Quando una cella ha zero sia per il giocatore che per l'avversario è vuota.


\begin{figure}[htbp]
  \centering
  \begin{subfigure}[b]{0.45\textwidth}
    \centering
    \scalebox{0.7}{
      \begin{othelloboard}{1}
        \dotmarkings
        \othelloarrayfirstrow   {0}{0}{0}{0}{0}{0}{0}{0}
        \othelloarraysecondrow  {0}{0}{0}{0}{0}{0}{0}{0}
        \othelloarraythirdrow   {0}{0}{0}{0}{0}{0}{0}{0}
        \othelloarrayfourthrow  {0}{0}{0}{1}{2}{0}{0}{0}
        \othelloarrayfifthrow   {0}{0}{0}{2}{1}{0}{0}{0}
        \othelloarraysixthrow   {0}{0}{0}{0}{0}{0}{0}{0}
        \othelloarrayseventhrow {0}{0}{0}{0}{0}{0}{0}{0}
        \othelloarrayeighthrow  {0}{0}{0}{0}{0}{0}{0}{0}
      \end{othelloboard}
    }
    \caption{Initial game position.}
  \end{subfigure}
  \hfill
  \begin{subfigure}[b]{0.45\textwidth}
    \centering
    \scalebox{0.7}{
      \begin{othelloboard}{1}
        \dotmarkings
        \othelloarrayfirstrow   {0}{0}{0}{0}{0}{0}{0}{0}
        \othelloarraysecondrow  {0}{0}{0}{0}{0}{0}{0}{0}
        \othelloarraythirdrow   {0}{0}{0}{0}{0}{0}{0}{0}
        \othelloarrayfourthrow  {0}{0}{0}{0}{0}{0}{0}{0}
        \othelloarrayfifthrow   {0}{0}{0}{0}{0}{0}{0}{0}
        \othelloarraysixthrow   {0}{0}{0}{0}{0}{0}{0}{0}
        \othelloarrayseventhrow {0}{0}{0}{0}{0}{0}{0}{0}
        \othelloarrayeighthrow  {0}{0}{0}{0}{0}{0}{0}{0}
        %
        \posannotation{a1}{\small 00}
        \posannotation{b1}{\small 01}
        \posannotation{c1}{\small 02}
        \posannotation{d1}{\small ...}
        \posannotation{a2}{\small 08}
        \posannotation{b2}{\small 09}
        \posannotation{c2}{\small ...}
        \posannotation{g8}{\small ...}
        \posannotation{h8}{\small 63}
      \end{othelloboard}
    }
    \caption{Rappresentazione della bitboard.}
  \end{subfigure}

  \vskip\baselineskip
  
  \begin{subfigure}[b]{0.45\textwidth}
    \centering
    \scalebox{0.7}{
      \begin{othelloboard}{1}
        \dotmarkings
        \othelloarrayfirstrow   {0}{0}{0}{0}{0}{0}{0}{0}
        \othelloarraysecondrow  {0}{0}{0}{0}{0}{0}{0}{0}
        \othelloarraythirdrow   {0}{0}{0}{0}{0}{0}{0}{0}
        \othelloarrayfourthrow  {0}{0}{0}{0}{0}{0}{0}{0}
        \othelloarrayfifthrow   {0}{0}{0}{0}{0}{0}{0}{0}
        \othelloarraysixthrow   {0}{0}{0}{0}{0}{0}{0}{0}
        \othelloarrayseventhrow {0}{0}{0}{0}{0}{0}{0}{0}
        \othelloarrayeighthrow  {0}{0}{0}{0}{0}{0}{0}{0}
        %
        \posannotation{a1}{\small 00}
        \posannotation{b1}{\small 01}
        \posannotation{c1}{\small 02}
        \posannotation{d1}{\small 03}
        \posannotation{e1}{\small 04}
        \posannotation{f1}{\small 05}
        \posannotation{g1}{\small 06}
        \posannotation{h1}{\small 07}
      \end{othelloboard}
    }
    \caption{EDGE pattern.}
  \end{subfigure}
  \hfill
  \begin{subfigure}[b]{0.45\textwidth}
    \centering
    \scalebox{0.7}{
      \begin{othelloboard}{1}
        \dotmarkings
        \othelloarrayfirstrow   {0}{0}{0}{0}{0}{0}{0}{0}
        \othelloarraysecondrow  {0}{0}{0}{0}{0}{0}{0}{0}
        \othelloarraythirdrow   {0}{0}{0}{0}{0}{0}{0}{0}
        \othelloarrayfourthrow  {0}{0}{0}{0}{0}{0}{0}{0}
        \othelloarrayfifthrow   {0}{0}{0}{0}{0}{0}{0}{0}
        \othelloarraysixthrow   {0}{0}{0}{0}{0}{0}{0}{0}
        \othelloarrayseventhrow {0}{0}{0}{0}{0}{0}{0}{0}
        \othelloarrayeighthrow  {0}{0}{0}{0}{0}{0}{0}{0}
        %
        \posannotation{a2}{\small 00}
        \posannotation{b2}{\small 01}
        \posannotation{c2}{\small 02}
        \posannotation{d2}{\small 03}
        \posannotation{e2}{\small 04}
        \posannotation{f2}{\small 05}
        \posannotation{g2}{\small 06}
        \posannotation{h2}{\small 07}
      \end{othelloboard}
    }
    \caption{R2 pattern.}
  \end{subfigure}

  \caption{Ordine dei bit per la definizione della bitboard e dei pattern.}
  \label{fig:reversi-bitboard-patterns}
\end{figure}

%
%  tlm_game_positions.tex
%
%  Copyright (c) 2026 Roberto Corradini. All rights reserved.
%
%  This file is part of the reversi program
%  http://github.com/rcrr/reversi
%
%  This program is free software; you can redistribute it and/or modify it
%  under the terms of the GNU General Public License as published by the
%  Free Software Foundation; either version 3, or (at your option) any
%  later version.
%
%  This program is distributed in the hope that it will be useful,
%  but WITHOUT ANY WARRANTY; without even the implied warranty of
%  MERCHANTABILITY or FITNESS FOR A PARTICULAR PURPOSE. See the
%  GNU General Public License for more details.
%
%  You should have received a copy of the GNU General Public License
%  along with this program; if not, write to the Free Software
%  Foundation, Inc., 59 Temple Place - Suite 330, Boston, MA  02111-1307, USA
%  or visit the site <http://www.gnu.org/licenses/>.
%


\subsection{Le posizioni (Game Positions)}

Ogni posizione è descritta dal \texttt{VALORE} dell'end-game ottenuto risolvendo la posizione stessa in maniera esatta con
apposito programma che applica l'algoritmo \texttt{MINIMAX} (o sue varianti sempre esatte ma più efficienti come \texttt{ALPHA-BETA}).

I \texttt{VALORI} possibili di una posizione sono i numeri interi pari compresi nell'intervallo \texttt{[-64..+64]}, pari alla differenza
di dischi dei due colori risultante alla fine della partita.
Zero significa pareggio, valori negativi una sconfitta, valori positivi una vittoria.

Le posizioni random, generate per costituire il dataset per il machine-learning (ML), sono archiviate in un database SQL chiamato REGAB (Reversi End-GAme Base).
Prendiamo una posizione risolta da REGAB con il comando SQL:

\begin{lstlisting}[language=SQL]
  SELECT seq, mover, opponent, empty_count, game_value, best_move FROM regab_prng_gp WHERE batch_id = 7 AND empty_count = 20 AND status = 'CMS' LIMIT 1;
\end{lstlisting}

{\tiny
\begin{verbatim}
   seq    |        mover        |       opponent       | empty_count | game_value | best_move 
----------+---------------------+----------------------+-------------+------------+-----------
 68230056 | 4611717676283199524 | -7855295674223658936 |          20 |         10 | F8
(1 row)
\end{verbatim}
}

Possiamo poi ricavare dal database ulteriori informazioni, come una vista 2D della posizione:
  
\begin{lstlisting}[language=SQL]
  SELECT game_position_pp_mop(mover, opponent, player) AS pos_68230056 FROM regab_prng_gp WHERE seq = 68230056;
\end{lstlisting}

{\tiny
\begin{verbatim}
     pos_68230056      
-----------------------
    a b c d e f g h   +
 1  . . @ O . @ O .   +
 2  . . . O O O O .   +
 3  . . O O O O @ .   +
 4  @ @ O @ O @ @ @   +
 5  . @ O @ O O @ @   +
 6  . O @ @ @ O O .   +
 7  . . O O O O O O   +
 8  . O . . O . @ O   +
 Player to move: BLACK
(1 row)
\end{verbatim}
}

Dove il carattere \texttt{'.'} significa cella vuota, \texttt{'@'} disco nero, e \texttt{'O'} disco bianco.
In questo caso il ``player to move'' è il \texttt{BLACK} che quindi è il \texttt{PLAYER}, mentre il giocatore bianco
è l' \texttt{OPPONENT}.

O una rappresentazione come stringa che poi possiamo usare come input per l'eseguibile che calcola il valore del game:

\begin{lstlisting}[language=SQL]
  SELECT game_position_to_string((mover, opponent, player)) AS pos_as_string_68230056 FROM regab_prng_gp WHERE seq = 68230056;
\end{lstlisting}

{\tiny
\begin{verbatim}
                      pos_as_string_68230056                       
-------------------------------------------------------------------
 ..bw.bw....wwww...wwwwb.bbwbwbbb.bwbwwbb.wbbbww...wwwwww.w..w.bwb
(1 row)
\end{verbatim}
}

Dove il carattere \texttt{'.'} significa cella vuota, \texttt{'b'} disco nero, \texttt{'w'} disco bianco. Il carattere più a sinistra
rappresenta la cella \texttt{00} (a1), muovendosi poi verso destra fino al carattere 64 che rappresenta la cella \texttt{63} (h8);
l'ultimo carattere a destra rappresenta invece il giocatore che deve muovere, \texttt{'b'} per \texttt{BLACK} e \texttt{'w'} per \texttt{WHITE}.

Eseguiamo il programma \texttt{endgame\_solver} dove il flag \texttt{-s es} richiama l' ``exact solver'':

\begin{lstlisting}[language=bash]
  ./build/bin/endgame_solver -s es -g ..bw.bw....wwww...wwwwb.bbwbwbbb.bwbwwbb.wbbbww...wwwwww.w..w.bwb
\end{lstlisting}

{\tiny
\begin{verbatim}
    a b c d e f g h 
 1  . . @ O . @ O . 
 2  . . . O O O O . 
 3  . . O O O O @ . 
 4  @ @ O @ O @ @ @ 
 5  . @ O @ O O @ @ 
 6  . O @ @ @ O O . 
 7  . . O O O O O O 
 8  . O . . O . @ O 
Player to move: BLACK

[node_count=45798972, leaf_count=8265202]
Final outcome: best move=F8, position value=10
\end{verbatim}
}

Il risultato è la posizione data da risolvere con il conto dei nodi e delle foglie del ``game-tree'' navigati dall'algoritmo.
Il programma, lanciato in batch mode, ha risolto e quindi taggato le posizioni del database REGAB con il valore esatto del game.

Ai fini del machine learning (ML) un record posizione di esempio con header è rappresentato come segue:

{\tiny
\begin{verbatim}
             PLAYER ;             OPPONENT ; GAME_VALUE
4611717676283199524 ; -7855295674223658936 ;         10
\end{verbatim}
}

Dove:

\begin{description}[leftmargin=*,widest=GAME\_VALUE,labelsep=1em]
\item[PLAYER] è la configurazione della board dei dischi occupati dal giocatore che deve muovere, in formato \texttt{int64} con segno.
  Il bit meno significativo è la casella \texttt{a1}, si procede per righe, come specificato sopra.
\item[OPPONENT] è la configurazione della board dei dischi occupati dall'avversario, stesso formato usato per \texttt{PLAYER}.
\item[GAME\_VALUE] è il valore esatto del game
\end{description}

Il valore del \texttt{GAME\_VALUE} è ottenuto con un software di soluzione esatta della posizione.
Il database contiene decine di milioni di record posizione risolti.
Le posizioni sono generate eseguendo delle mosse legali scelte a random a partire dalle condizione iniziali.

\include{tlm_definizioni_python}
\include{tlm_geometria_della_board}
\include{tlm_teoria_dei_pattern}
%
%  tlm_definizione_e_calcolo_dei_pattern.tex
%
%  Copyright (c) 2026 Roberto Corradini. All rights reserved.
%
%  This file is part of the reversi program
%  http://github.com/rcrr/reversi
%
%  This program is free software; you can redistribute it and/or modify it
%  under the terms of the GNU General Public License as published by the
%  Free Software Foundation; either version 3, or (at your option) any
%  later version.
%
%  This program is distributed in the hope that it will be useful,
%  but WITHOUT ANY WARRANTY; without even the implied warranty of
%  MERCHANTABILITY or FITNESS FOR A PARTICULAR PURPOSE. See the
%  GNU General Public License for more details.
%
%  You should have received a copy of the GNU General Public License
%  along with this program; if not, write to the Free Software
%  Foundation, Inc., 59 Temple Place - Suite 330, Boston, MA  02111-1307, USA
%  or visit the site <http://www.gnu.org/licenses/>.
%


\section{Definizione e Calcolo dei Pattern}

Un \texttt{PATTERN} è un insieme di celle.
Ad esempio \texttt{EDGE} è il nome assegnato alle otto caselle di un bordo della board. Ci sono quattro EDGE in una board,
ottenuti ruotando di 0, 90, 180, 270 gradi il pattern di base, per convenzione quello sul bordo a nord.

Il pattern \texttt{EDGE} è quindi definito dalle caselle \texttt{\{A1, B1, C1, D1, E1, F1, G1, H1\}}, la seconda istanza del pattern è situata
sul bordo est \texttt{\{H1, H2, H3, H4, H5, H6, H7, H8\}}, la terza sul bordo sud \texttt{\{H8, G8, F8, E8, D8, C8, B8, A8\}},
la quarta su quello ovest \texttt{\{A8, A7, A6, A5, A4, A3, A2, A1\}}.
Le quattro istanze del pattern possono essere portate sulla rappresentazione di base applicando le quattro rotazioni antiorarie.
Dalla configurazione di base invece posso riportare il pattern alla sua istanza con le rotazioni orarie corrispondenti.
Il pattern può anche essere definito con la maschera dei bit sulla bitboard, uno \texttt{SquareSet} quindi che
rappresentato in esadecimale vale \texttt{0x00000000000000FF}.

Un secondo pattern che prendiamo in considerazione è \texttt{R2}, definito in esadecimale come \texttt{0x000000000000FF00}.

I pattern sono potenzialmente tantissimi, per la massima generalità focalizziamoci su un pattern che copra tutte le casistiche di queste
astrazioni ma che sia al contempo semplice, lo definiamo con il set di quattro elementi \texttt{\{A1, B1, C1, A2\}} equivalente alla maschera
\texttt{0x0000000000000107} e lo chiamiamo \texttt{ELLE}.

% WE ARE HERE

I pattern sono potenzialmente tantissimi, ma per ora soffermiamoci su questi due \texttt{EDGE} ed \texttt{R2}, questi infatti sono sufficienti
ad introdurre tutta la complessità che dobbiamo affrontare, senza perdere di generalità.

Il pattern \texttt{EDGE}, ed analogamente \texttt{R2}, è caratterizzato da otto celle.
Ogni cella ha tre configurazioni possibili: vuota, occupata da un disco nero, occupata da un disco bianco.
Diamoci la convenzione che \texttt{BLACK} significa \texttt{PLAYER}, e \texttt{WHITE} \texttt{OPPONENT}.
Le configurazioni che un \texttt{EDGE} può assumere sono quindi tre elevato alla otto quindi $3^8 = 6561$, seimilacinquecentosessantuno.
Dobbiamo definire un encoding di queste configurazioni, in maniera da assegnargli un nome univoco.
Questo nome lo chiamiamo \texttt{INDEX}.

Per calcolare il valore dell'indice $i$ nell'equazione~\ref{eq:calcolo_indice}, quindi procediamo come segue.
Ognuno dei tre colori che può assumere una cella ha un valore $c$ assegnato.
Assegniamo il valore zero alla cella vuota, il valore uno alla cella occupata dal nero, ed il valore due a quella occupata dal bianco.
Assegniamo un ordine univoco alle celle di un pattern, dalla meno significativa alla più significativa, e chiamiamo $p$ l'ordinale di questa posizione.
Quindi per l'\texttt{EDGE} la casella \texttt{A1} è la zero, \texttt{B1} è la uno, e la \texttt{H1} è la sette, che è l'ultima e quindi la $n-esima$.
L'indice di una configurazione è realizzato sommando il contributo di ciascuna cella del pattern,
con valore $[0..2]$ a secondo del colore (empty=0, black=1, white=2)
moltiplicato per tre elevato all'esponente ottenuto dalla posizione ordinale $p$ della cella nel pattern.


\begin{equation}
  i = \sum_{p=0}^{n} c \cdot 3^p
  \label{eq:calcolo_indice}
\end{equation}

Per colcolare i valori dei quattro indici del pattern \texttt{EDGE}, ma anologamente poi per il pattern \texttt{R2} od ogni altro pattern, si procede con questa sequenza
di operazioni. Viene prima ruotata la board, poi si esegue l'operazione di mask e di pack, quindi viene calcolato l'indice con la~\ref{eq:calcolo_indice}.

Per la massima generalità le istanze di un pattern nella bord possono essere fino ad otto, questo nel caso in cui il pattern non abbia una simmetria rispetto all'asse verticale
della board. In questo esempio basato sui due pattern \texttt{EDGE} ed \texttt{R2} la condizione non si presenta, non ci espone comunque questa semplificazione a perdere di generalità
in tutti i passaggi seguenti per la definizione del modello matematico della funzione di valutazione del valore del gioco.

\begin{lstlisting}[language=C]

  SquareSet
  board_pattern_pack_edge (SquareSet s)
  {
    return s & 0x00000000000000ff;
  }

  SquareSet
  board_pattern_unpack_edge (SquareSet s)
  {
    return s & 0x00000000000000ff;
  }

  SquareSet
  board_pattern_pack_r2 (SquareSet s)
  {
    return (s >> 8) & 0x00000000000000ff;
  }

  SquareSet
  board_pattern_unpack_r2 (SquareSet s)
  {
    return (s & 0x00000000000000ff) << 8;
  }
  
\end{lstlisting}


\begin{lstlisting}[language=C]

  void
  board_pattern_compute_indexes (board_pattern_index_t *indexes,
                                 const board_pattern_t *const p,
                                 const board_t *const b)
  {
    board_t tb[8];
    SquareSet m, o, c;
    board_trans_f tf;
    SquareSet mover_pattern_packed[8];
    SquareSet opponent_pattern_packed[8];

    /*                       0  1   2   3    4    5     6     7      8      9      10      11      12        13       14        15 */
    const uint64_t cim[] = { 1, 3,  9, 27,  81, 243,  729, 2187,  6561, 19683,  59049, 177147,  531441, 1594323, 4782969, 14348907 };
    const uint64_t cio[] = { 2, 6, 18, 54, 162, 486, 1458, 4374, 13122, 39366, 118098, 354294, 1062882, 3188646, 9565938, 28697814 };

    /*
    * This part should go out of the specific pattern, and done once for all patterns.
    * For now we focus on EDGE pattern alone.
    */

    m = board_get_mover_square_set(b);
    o = board_get_opponent_square_set(b);

    for (int i = 0; i < p->n_instances; i++) {
      tf = p->trans_to_principal_f[i];
      board_set_square_sets(tb + i, tf(m), tf(o));
    }

    for (int i = 0; i < p->n_instances; i++) {
      mover_pattern_packed[i] = p->pattern_pack_f(board_get_mover_square_set(tb + i));
      opponent_pattern_packed[i] = p->pattern_pack_f(board_get_opponent_square_set(tb + i));
    }

    for (int i = 0; i < p->n_instances; i++) {
      indexes[i] = 0;
      for (int j = 0; j < p->n_squares; j++) {
        c = 1ULL << j;
        indexes[i] += (c & mover_pattern_packed[i]   ) ? cim[j] : 0;
        indexes[i] += (c & opponent_pattern_packed[i]) ? cio[j] : 0;
      }
    }
    
\end{lstlisting}


\include{tlm_evaluation_function}

\appendix

%
%  tlm_appendix_catalogo_dei_pattern.tex
%
%  Copyright (c) 2026 Roberto Corradini. All rights reserved.
%
%  This file is part of the reversi program
%  http://github.com/rcrr/reversi
%
%  This program is free software; you can redistribute it and/or modify it
%  under the terms of the GNU General Public License as published by the
%  Free Software Foundation; either version 3, or (at your option) any
%  later version.
%
%  This program is distributed in the hope that it will be useful,
%  but WITHOUT ANY WARRANTY; without even the implied warranty of
%  MERCHANTABILITY or FITNESS FOR A PARTICULAR PURPOSE. See the
%  GNU General Public License for more details.
%
%  You should have received a copy of the GNU General Public License
%  along with this program; if not, write to the Free Software
%  Foundation, Inc., 59 Temple Place - Suite 330, Boston, MA  02111-1307, USA
%  or visit the site <http://www.gnu.org/licenses/>.
%

\captionsetup[subfigure]{justification=centering} % Forza tutte le subcaption a essere centrate

\newrobustcmd{\hexmask}[1]{%
    \\[0.5ex] % Va a capo
    \makebox[\textwidth][c]{% Centra perfettamente
        \fontsize{6}{7.2}\selectfont\texttt{0x#1}%
    }
}
  
\section{Catalogo dei Pattern}

In questa appendice vengono visualizzati i pattern principali utilizzati dal sistema. Per ogni pattern sono mostrate le 8 trasformazioni del gruppo $D_4$. I numeri all'interno delle celle indicano l'indice di packing progressivo (0, 1, 2, ...) all'interno del pattern, permettendo di tracciare visivamente la permutazione delle celle sotto ogni trasformazione.
\newpage

% Inizia il ciclo per ogni riga del CSV mdp.csv
\DTLforeach{master_data_patterns}{
  \mdpName=name,
  \mdpMask=mask}{
  
  \subsection{Pattern \mdpName}
  [Pattern: name = \mdpName, mask = \mdpMask]
  
  %\begin{lstlisting}[language=bash, basicstyle=\ttfamily\tiny]
  %\end{lstlisting}
  \begin{tcolorbox}[sharp corners, fontupper=\ttfamily\tiny, colback=gray!10]
    \mdpMask
  \end{tcolorbox}
  \newpage
}

\subsection{Pattern ELLE}

\begin{lstlisting}[language=bash, basicstyle=\ttfamily\tiny]
[Pattern: name = ELLE, mask = 0x0000000000000107]
  [n_squares = 4, n_configurations = 81, n_instances = 8, n_stabilizer = 1]
  Cells:               [A1, B1, C1, A2]
  Transformed masks:   
    0x0000000000000107, 0x00000000008080C0, 0xE080000000000000, 0x0301010000000000, 
    0x00000000000080E0, 0xC080800000000000, 0x0701000000000000, 0x0000000000010103
  Mask indexes:        [0, 1, 2, 3, 4, 5, 6, 7]
  Unique masks:        
    0x0000000000000107, 0x00000000008080C0, 0xE080000000000000, 0x0301010000000000, 
    0x00000000000080E0, 0xC080800000000000, 0x0701000000000000, 0x0000000000010103
  Unique mask indexes: [0, 1, 2, 3, 4, 5, 6, 7]
  Transf. functions:   [Id, Rot90, Vert, DiagP]
  Anti-transf. f.:     [Id, Rot270, Vert, DiagP]
  Symmetry functions:  []
\end{lstlisting}

\begin{figure}[htbp]
    \centering
    \begin{subfigure}[b]{0.23\textwidth}
        \centering
        \scalebox{0.45}{
            \begin{othelloboard}{1}
                \posannotation{a1}{0} \posannotation{b1}{1} \posannotation{c1}{2} \posannotation{a2}{3}
            \end{othelloboard}
        }
        %\caption{$T_0$ : Id \\ \fontsize{5}{6}\selectfont \texttt{0000000000000107}}
        \caption{$T_0$ : Id \hexmask{0000000000000107}}
    \end{subfigure}
    \hfill
    \begin{subfigure}[b]{0.23\textwidth}
        \centering
        \scalebox{0.45}{
            \begin{othelloboard}{1}
                \posannotation{h1}{0} \posannotation{h2}{1} \posannotation{h3}{2} \posannotation{g1}{3}
            \end{othelloboard}
        }
        \caption{$T_1$ : Rot90 \hexmask{00000000008080C0}}
    \end{subfigure}
    \hfill
    \begin{subfigure}[b]{0.23\textwidth}
        \centering
        \scalebox{0.45}{
            \begin{othelloboard}{1}
                \posannotation{h8}{0} \posannotation{g8}{1} \posannotation{f8}{2} \posannotation{h7}{3}
            \end{othelloboard}
        }
        \caption{$T_2$ : Rot180 \hexmask{E080000000000000}}
    \end{subfigure}
    \hfill
    \begin{subfigure}[b]{0.23\textwidth}
        \centering
        \scalebox{0.45}{
            \begin{othelloboard}{1}
                \posannotation{a8}{0} \posannotation{a7}{1} \posannotation{a6}{2} \posannotation{b8}{3}
            \end{othelloboard}
        }
        \caption{$T_3$ : Rot270 \hexmask{0301010000000000}}
    \end{subfigure}

    \vspace{0.3cm}

    \begin{subfigure}[b]{0.23\textwidth}
        \centering
        \scalebox{0.45}{
            \begin{othelloboard}{1}
                \posannotation{h1}{0} \posannotation{g1}{1} \posannotation{f1}{2} \posannotation{h2}{3}
            \end{othelloboard}
        }
        \caption{$T_4$ : Vert \hexmask{00000000000080E0}}
    \end{subfigure}
    \hfill
    \begin{subfigure}[b]{0.23\textwidth}
        \centering
        \scalebox{0.45}{
            \begin{othelloboard}{1}
                \posannotation{h8}{0} \posannotation{h7}{1} \posannotation{h6}{2} \posannotation{g8}{3}
            \end{othelloboard}
        }
        \caption{$T_5$ : DiagP \hexmask{C080800000000000}}
    \end{subfigure}
    \hfill
    \begin{subfigure}[b]{0.23\textwidth}
        \centering
        \scalebox{0.45}{
            \begin{othelloboard}{1}
                \posannotation{a8}{0} \posannotation{b8}{1} \posannotation{c8}{2} \posannotation{a7}{3}
            \end{othelloboard}
        }
        \caption{$T_6$ : Horiz \hexmask{0701000000000000}}
    \end{subfigure}
    \hfill
    \begin{subfigure}[b]{0.23\textwidth}
        \centering
        \scalebox{0.45}{
            \begin{othelloboard}{1}
                \posannotation{a1}{0} \posannotation{a2}{1} \posannotation{a3}{2} \posannotation{b1}{3}
            \end{othelloboard}
        }
        \caption{$T_7$ : DiagA \hexmask{0000000000010103}}
    \end{subfigure}
    \caption{Istanze e indici di permutazione del pattern ELLE.}
\end{figure}

\newpage
\subsection{Pattern EDGE}

\begin{lstlisting}[language=bash, basicstyle=\ttfamily\tiny]
[Pattern: name = EDGE, mask = 0x00000000000000ff]
  [n_squares = 8, n_configurations = 6561, n_instances = 4, n_stabilizer = 2]
  Cells:               [A1, B1, C1, D1, E1, F1, G1, H1]
  Transformed masks:   
    0x00000000000000FF, 0x8080808080808080, 0xFF00000000000000, 0x0101010101010101, 
    0x00000000000000FF, 0x8080808080808080, 0xFF00000000000000, 0x0101010101010101
  Mask indexes:        [0, 1, 2, 3, 0, 1, 2, 3]
  Unique masks:        
    0x00000000000000FF, 0x8080808080808080, 0xFF00000000000000, 0x0101010101010101
  Unique mask indexes: [0, 1, 2, 3]
  Transf. functions:   [Id, Rot90, Rot180, Rot270]
  Anti-transf. f.:     [Id, Rot270, Rot180, Rot90]
  Symmetry functions:  [Vert]
\end{lstlisting}

\begin{figure}[htbp]
    \centering
    \begin{subfigure}[b]{0.23\textwidth}
        \centering
        \scalebox{0.45}{
            \begin{othelloboard}{1}
                \posannotation{a1}{0} \posannotation{b1}{1} \posannotation{c1}{2} \posannotation{d1}{3} \posannotation{e1}{4} \posannotation{f1}{5} \posannotation{g1}{6} \posannotation{h1}{7}
            \end{othelloboard}
        }
        \caption{$T_0$ : Id \\ \fontsize{5}{6}\selectfont \texttt{00000000000000FF}}
    \end{subfigure}
    \hfill
    \begin{subfigure}[b]{0.23\textwidth}
        \centering
        \scalebox{0.45}{
            \begin{othelloboard}{1}
                \posannotation{h1}{0} \posannotation{h2}{1} \posannotation{h3}{2} \posannotation{h4}{3} \posannotation{h5}{4} \posannotation{h6}{5} \posannotation{h7}{6} \posannotation{h8}{7}
            \end{othelloboard}
        }
        \caption{$T_1$ : Rot90 \\ \fontsize{5}{6}\selectfont \texttt{8080808080808080}}
    \end{subfigure}
    \hfill
    \begin{subfigure}[b]{0.23\textwidth}
        \centering
        \scalebox{0.45}{
            \begin{othelloboard}{1}
                \posannotation{h8}{0} \posannotation{g8}{1} \posannotation{f8}{2} \posannotation{e8}{3} \posannotation{d8}{4} \posannotation{c8}{5} \posannotation{b8}{6} \posannotation{a8}{7}
            \end{othelloboard}
        }
        \caption{$T_2$ : Rot180 \\ \fontsize{5}{6}\selectfont \texttt{FF00000000000000}}
    \end{subfigure}
    \hfill
    \begin{subfigure}[b]{0.23\textwidth}
        \centering
        \scalebox{0.45}{
            \begin{othelloboard}{1}
                \posannotation{a8}{0} \posannotation{a7}{1} \posannotation{a6}{2} \posannotation{a5}{3} \posannotation{a4}{4} \posannotation{a3}{5} \posannotation{a2}{6} \posannotation{a1}{7}
            \end{othelloboard}
        }
        \caption{$T_3$ : Rot270 \\ \fontsize{5}{6}\selectfont \texttt{0101010101010101}}
    \end{subfigure}

    \vspace{0.3cm}

    \begin{subfigure}[b]{0.23\textwidth}
        \centering
        \scalebox{0.45}{
            \begin{othelloboard}{1}
                \posannotation{h1}{0} \posannotation{g1}{1} \posannotation{f1}{2} \posannotation{e1}{3} \posannotation{d1}{4} \posannotation{c1}{5} \posannotation{b1}{6} \posannotation{a1}{7}
            \end{othelloboard}
        }
        \caption{$T_4$ : Vert \\ \fontsize{5}{6}\selectfont \texttt{00000000000000FF}}
    \end{subfigure}
    \hfill
    \begin{subfigure}[b]{0.23\textwidth}
        \centering
        \scalebox{0.45}{
            \begin{othelloboard}{1}
                \posannotation{h8}{0} \posannotation{h7}{1} \posannotation{h6}{2} \posannotation{h5}{3} \posannotation{h4}{4} \posannotation{h3}{5} \posannotation{h2}{6} \posannotation{h1}{7}
            \end{othelloboard}
        }
        \caption{$T_5$ : DiagP \\ \fontsize{5}{6}\selectfont \texttt{8080808080808080}}
    \end{subfigure}
    \hfill
    \begin{subfigure}[b]{0.23\textwidth}
        \centering
        \scalebox{0.45}{
            \begin{othelloboard}{1}
                \posannotation{a8}{0} \posannotation{b8}{1} \posannotation{c8}{2} \posannotation{d8}{3} \posannotation{e8}{4} \posannotation{f8}{5} \posannotation{g8}{6} \posannotation{h8}{7}
            \end{othelloboard}
        }
        \caption{$T_6$ : Horiz \\ \fontsize{5}{6}\selectfont \texttt{FF00000000000000}}
    \end{subfigure}
    \hfill
    \begin{subfigure}[b]{0.23\textwidth}
        \centering
        \scalebox{0.45}{
            \begin{othelloboard}{1}
                \posannotation{a1}{0} \posannotation{a2}{1} \posannotation{a3}{2} \posannotation{a4}{3} \posannotation{a5}{4} \posannotation{a6}{5} \posannotation{a7}{6} \posannotation{a8}{7}
            \end{othelloboard}
        }
        \caption{$T_7$ : DiagA \\ \fontsize{5}{6}\selectfont \texttt{0101010101010101}}
    \end{subfigure}
    \caption{Istanze e indici di permutazione del pattern EDGE.}
\end{figure}

\newpage
\subsection{Pattern ZSHAPE}

\begin{lstlisting}[language=bash, basicstyle=\ttfamily\tiny]
[Pattern: name = ZSHAPE, mask = 0x0000000c30000000]
  [n_squares = 4, n_configurations = 81, n_instances = 4, n_stabilizer = 2]
  Cells:               [E4, F4, C5, D5]
  Transformed masks:   
    0x0000000C30000000, 0x0000101008080000, 0x0000000C30000000, 0x0000101008080000, 
    0x000000300C000000, 0x0000080810100000, 0x000000300C000000, 0x0000080810100000
  Mask indexes:        [0, 1, 0, 1, 4, 5, 4, 5]
  Unique masks:        
    0x0000000C30000000, 0x0000101008080000, 0x000000300C000000, 0x0000080810100000
  Unique mask indexes: [0, 1, 4, 5]
  Transf. functions:   [Id, Rot90, Vert, DiagP]
  Anti-transf. f.:     [Id, Rot270, Vert, DiagP]
  Symmetry functions:  [Rot180]
\end{lstlisting}

\begin{figure}[htbp]
    \centering
    \begin{subfigure}[b]{0.23\textwidth}
        \centering
        \scalebox{0.45}{
            \begin{othelloboard}{1}
                \posannotation{e4}{0} \posannotation{f4}{1} \posannotation{c5}{2} \posannotation{d5}{3}
            \end{othelloboard}
        }
        \caption{$T_0$ : Id \\ \fontsize{5}{6}\selectfont \texttt{0000000C30000000}}
    \end{subfigure}
    \hfill
    \begin{subfigure}[b]{0.23\textwidth}
        \centering
        \scalebox{0.45}{
            \begin{othelloboard}{1}
                \posannotation{e5}{0} \posannotation{e6}{1} \posannotation{d3}{2} \posannotation{d4}{3}
            \end{othelloboard}
        }
        \caption{$T_1$ : Rot90 \\ \fontsize{5}{6}\selectfont \texttt{0000101008080000}}
    \end{subfigure}
    \hfill
    \begin{subfigure}[b]{0.23\textwidth}
        \centering
        \scalebox{0.45}{
            \begin{othelloboard}{1}
                \posannotation{d5}{0} \posannotation{c5}{1} \posannotation{f4}{2} \posannotation{e4}{3}
            \end{othelloboard}
        }
        \caption{$T_2$ : Rot180 \\ \fontsize{5}{6}\selectfont \texttt{0000000C30000000}}
    \end{subfigure}
    \hfill
    \begin{subfigure}[b]{0.23\textwidth}
        \centering
        \scalebox{0.45}{
            \begin{othelloboard}{1}
                \posannotation{d4}{0} \posannotation{d3}{1} \posannotation{e6}{2} \posannotation{e5}{3}
            \end{othelloboard}
        }
        \caption{$T_3$ : Rot270 \\ \fontsize{5}{6}\selectfont \texttt{0000101008080000}}
    \end{subfigure}

    \vspace{0.3cm}

    \begin{subfigure}[b]{0.23\textwidth}
        \centering
        \scalebox{0.45}{
            \begin{othelloboard}{1}
                \posannotation{d4}{0} \posannotation{c4}{1} \posannotation{f5}{2} \posannotation{e5}{3}
            \end{othelloboard}
        }
        \caption{$T_4$ : Vert \\ \fontsize{5}{6}\selectfont \texttt{000000300C000000}}
    \end{subfigure}
    \hfill
    \begin{subfigure}[b]{0.23\textwidth}
        \centering
        \scalebox{0.45}{
            \begin{othelloboard}{1}
                \posannotation{e4}{0} \posannotation{e3}{1} \posannotation{d6}{2} \posannotation{d5}{3}
            \end{othelloboard}
        }
        \caption{$T_5$ : DiagP \\ \fontsize{5}{6}\selectfont \texttt{0000080810100000}}
    \end{subfigure}
    \hfill
    \begin{subfigure}[b]{0.23\textwidth}
        \centering
        \scalebox{0.45}{
            \begin{othelloboard}{1}
                \posannotation{e5}{0} \posannotation{f5}{1} \posannotation{c4}{2} \posannotation{d4}{3}
            \end{othelloboard}
        }
        \caption{$T_6$ : Horiz \\ \fontsize{5}{6}\selectfont \texttt{000000300C000000}}
    \end{subfigure}
    \hfill
    \begin{subfigure}[b]{0.23\textwidth}
        \centering
        \scalebox{0.45}{
            \begin{othelloboard}{1}
                \posannotation{d5}{0} \posannotation{d6}{1} \posannotation{e3}{2} \posannotation{e4}{3}
            \end{othelloboard}
        }
        \caption{$T_7$ : DiagA \\ \fontsize{5}{6}\selectfont \texttt{0000080810100000}}
    \end{subfigure}
    \caption{Istanze e indici di permutazione del pattern ZSHAPE.}
\end{figure}

\newpage
\subsection{Pattern RCT2X4}

\begin{lstlisting}[language=bash, basicstyle=\ttfamily\tiny]
[Pattern: name = RCT2X4, mask = 0x0000003c3c000000]
  [n_squares = 8, n_configurations = 6561, n_instances = 2, n_stabilizer = 4]
  Cells:               [C4, D4, E4, F4, C5, D5, E5, F5]
  Transformed masks:   
    0x0000003C3C000000, 0x0000181818180000, 0x0000003C3C000000, 0x0000181818180000, 
    0x0000003C3C000000, 0x0000181818180000, 0x0000003C3C000000, 0x0000181818180000
  Mask indexes:        [0, 1, 0, 1, 0, 1, 0, 1]
  Unique masks:        
    0x0000003C3C000000, 0x0000181818180000
  Unique mask indexes: [0, 1]
  Transf. functions:   [Id, Rot90]
  Anti-transf. f.:     [Id, Rot270]
  Symmetry functions:  [Rot180, Vert, Horiz]
\end{lstlisting}

\begin{figure}[htbp]
    \centering
    \begin{subfigure}[b]{0.23\textwidth}
        \centering
        \scalebox{0.45}{
            \begin{othelloboard}{1}
                \posannotation{c4}{0} \posannotation{d4}{1} \posannotation{e4}{2} \posannotation{f4}{3} 
                \posannotation{c5}{4} \posannotation{d5}{5} \posannotation{e5}{6} \posannotation{f5}{7}
            \end{othelloboard}
        }
        \caption{$T_0$ : Id \\ \fontsize{5}{6}\selectfont \texttt{0000003C3C000000}}
    \end{subfigure}
    \hfill
    \begin{subfigure}[b]{0.23\textwidth}
        \centering
        \scalebox{0.45}{
            \begin{othelloboard}{1}
                \posannotation{e3}{0} \posannotation{e4}{1} \posannotation{e5}{2} \posannotation{e6}{3} 
                \posannotation{d3}{4} \posannotation{d4}{5} \posannotation{d5}{6} \posannotation{d6}{7}
            \end{othelloboard}
        }
        \caption{$T_1$ : Rot90 \\ \fontsize{5}{6}\selectfont \texttt{0000181818180000}}
    \end{subfigure}
    \hfill
    \begin{subfigure}[b]{0.23\textwidth}
        \centering
        \scalebox{0.45}{
            \begin{othelloboard}{1}
                \posannotation{f5}{0} \posannotation{e5}{1} \posannotation{d5}{2} \posannotation{c5}{3} 
                \posannotation{f4}{4} \posannotation{e4}{5} \posannotation{d4}{6} \posannotation{c4}{7}
            \end{othelloboard}
        }
        \caption{$T_2$ : Rot180 \\ \fontsize{5}{6}\selectfont \texttt{0000003C3C000000}}
    \end{subfigure}
    \hfill
    \begin{subfigure}[b]{0.23\textwidth}
        \centering
        \scalebox{0.45}{
            \begin{othelloboard}{1}
                \posannotation{d6}{0} \posannotation{d5}{1} \posannotation{d4}{2} \posannotation{d3}{3} 
                \posannotation{e6}{4} \posannotation{e5}{5} \posannotation{e4}{6} \posannotation{e3}{7}
            \end{othelloboard}
        }
        \caption{$T_3$ : Rot270 \\ \fontsize{5}{6}\selectfont \texttt{0000181818180000}}
    \end{subfigure}

    \vspace{0.3cm}

    \begin{subfigure}[b]{0.23\textwidth}
        \centering
        \scalebox{0.45}{
            \begin{othelloboard}{1}
                \posannotation{f4}{0} \posannotation{e4}{1} \posannotation{d4}{2} \posannotation{c4}{3} 
                \posannotation{f5}{4} \posannotation{e5}{5} \posannotation{d5}{6} \posannotation{c5}{7}
            \end{othelloboard}
        }
        \caption{$T_4$ : Vert \\ \fontsize{5}{6}\selectfont \texttt{0000003C3C000000}}
    \end{subfigure}
    \hfill
    \begin{subfigure}[b]{0.23\textwidth}
        \centering
        \scalebox{0.45}{
            \begin{othelloboard}{1}
                \posannotation{e6}{0} \posannotation{e5}{1} \posannotation{e4}{2} \posannotation{e3}{3} 
                \posannotation{d6}{4} \posannotation{d5}{5} \posannotation{d4}{6} \posannotation{d3}{7}
            \end{othelloboard}
        }
        \caption{$T_5$ : DiagP \\ \fontsize{5}{6}\selectfont \texttt{0000181818180000}}
    \end{subfigure}
    \hfill
    \begin{subfigure}[b]{0.23\textwidth}
        \centering
        \scalebox{0.45}{
            \begin{othelloboard}{1}
                \posannotation{c5}{0} \posannotation{d5}{1} \posannotation{e5}{2} \posannotation{f5}{3} 
                \posannotation{c4}{4} \posannotation{d4}{5} \posannotation{e4}{6} \posannotation{f4}{7}
            \end{othelloboard}
        }
        \caption{$T_6$ : Horiz \\ \fontsize{5}{6}\selectfont \texttt{0000003C3C000000}}
    \end{subfigure}
    \hfill
    \begin{subfigure}[b]{0.23\textwidth}
        \centering
        \scalebox{0.45}{
            \begin{othelloboard}{1}
                \posannotation{d3}{0} \posannotation{d4}{1} \posannotation{d5}{2} \posannotation{d6}{3} 
                \posannotation{e3}{4} \posannotation{e4}{5} \posannotation{e5}{6} \posannotation{e6}{7}
            \end{othelloboard}
        }
        \caption{$T_7$ : DiagA \\ \fontsize{5}{6}\selectfont \texttt{0000181818180000}}
    \end{subfigure}
    \caption{Istanze e indici di permutazione del pattern RCT2X4.}
\end{figure}

\newpage
\subsection{Pattern BARBEL}

\begin{lstlisting}[language=bash, basicstyle=\ttfamily\tiny]
[Pattern: name = BARBEL, mask = 0x030304081020C0C0]
  [n_squares = 12, n_configurations = 531441, n_instances = 2, n_stabilizer = 4]
  Cells:               [G1, H1, G2, H2, F3, E4, D5, C6, A7, B7, A8, B8]
  Transformed masks:   
    0x030304081020C0C0, 0xC0C0201008040303, 0x030304081020C0C0, 0xC0C0201008040303,
    0xC0C0201008040303, 0x030304081020C0C0, 0xC0C0201008040303, 0x030304081020C0C0
  Mask indexes:        [0, 1, 0, 1, 1, 0, 1, 0]
  Unique masks:        
    0x030304081020C0C0, 0xC0C0201008040303
  Unique mask indexes: [0, 1]
  Transf. functions:   [Id, Rot90]
  Anti-transf. f.:     [Id, Rot270]
  Symmetry functions:  [Rot180, DiagP, DiagA]
\end{lstlisting}

\begin{figure}[htbp]
    \centering
    \begin{subfigure}[b]{0.23\textwidth}
        \centering
        \scalebox{0.45}{
            \begin{othelloboard}{1}
                \posannotation{g1}{0} \posannotation{h1}{1} \posannotation{g2}{2} \posannotation{h2}{3}
                \posannotation{f3}{4} \posannotation{e4}{5} \posannotation{d5}{6} \posannotation{c6}{7}
                \posannotation{a7}{8} \posannotation{b7}{9} \posannotation{a8}{10} \posannotation{b8}{11}
            \end{othelloboard}
        }
        \caption{$T_0$ : Id \\ \fontsize{5}{6}\selectfont \texttt{030304081020C0C0}}
    \end{subfigure}
    \hfill
    \begin{subfigure}[b]{0.23\textwidth}
        \centering
        \scalebox{0.45}{
            \begin{othelloboard}{1}
                \posannotation{a1}{10} \posannotation{b1}{8} \posannotation{a2}{11} \posannotation{b2}{9}
                \posannotation{c3}{7} \posannotation{d4}{6} \posannotation{e5}{5} \posannotation{f6}{4}
                \posannotation{g7}{2} \posannotation{h7}{0} \posannotation{g8}{3} \posannotation{h8}{1}
            \end{othelloboard}
        }
        \caption{$T_1$ : Rot90 \\ \fontsize{5}{6}\selectfont \texttt{C0C0201008040303}}
    \end{subfigure}
    \hfill
    \begin{subfigure}[b]{0.23\textwidth}
        \centering
        \scalebox{0.45}{
            \begin{othelloboard}{1}
                \posannotation{g1}{11} \posannotation{h1}{10} \posannotation{g2}{9} \posannotation{h2}{8}
                \posannotation{f3}{7} \posannotation{e4}{6} \posannotation{d5}{5} \posannotation{c6}{4}
                \posannotation{a7}{3} \posannotation{b7}{2} \posannotation{a8}{1} \posannotation{b8}{0}
            \end{othelloboard}
        }
        \caption{$T_2$ : Rot180 \\ \fontsize{5}{6}\selectfont \texttt{030304081020C0C0}}
    \end{subfigure}
    \hfill
    \begin{subfigure}[b]{0.23\textwidth}
        \centering
        \scalebox{0.45}{
            \begin{othelloboard}{1}
                \posannotation{a1}{1} \posannotation{b1}{3} \posannotation{a2}{0} \posannotation{b2}{2}
                \posannotation{c3}{4} \posannotation{d4}{5} \posannotation{e5}{6} \posannotation{f6}{7}
                \posannotation{g7}{9} \posannotation{h7}{11} \posannotation{g8}{8} \posannotation{h8}{10}
            \end{othelloboard}
        }
        \caption{$T_3$ : Rot270 \\ \fontsize{5}{6}\selectfont \texttt{C0C0201008040303}}
    \end{subfigure}

    \vspace{0.3cm}
    
    \begin{subfigure}[b]{0.23\textwidth}
        \centering
        \scalebox{0.45}{
            \begin{othelloboard}{1}
                \posannotation{a1}{1} \posannotation{b1}{0} \posannotation{a2}{3} \posannotation{b2}{2}
                \posannotation{c3}{4} \posannotation{d4}{5} \posannotation{e5}{6} \posannotation{f6}{7}
                \posannotation{g7}{9} \posannotation{h7}{8} \posannotation{g8}{11} \posannotation{h8}{10}
            \end{othelloboard}
        }
        \caption{$T_4$ : Vert \\ \fontsize{5}{6}\selectfont \texttt{C0C0201008040303}}
    \end{subfigure}
    \hfill
    \begin{subfigure}[b]{0.23\textwidth}
        \centering
        \scalebox{0.45}{
            \begin{othelloboard}{1}
                \posannotation{g1}{3} \posannotation{h1}{1} \posannotation{g2}{2} \posannotation{h2}{0}
                \posannotation{f3}{4} \posannotation{e4}{5} \posannotation{d5}{6} \posannotation{c6}{7}
                \posannotation{a7}{11} \posannotation{b7}{9} \posannotation{a8}{10} \posannotation{b8}{8}
            \end{othelloboard}
        }
        \caption{$T_5$ : DiagP \\ \fontsize{5}{6}\selectfont \texttt{030304081020C0C0}}
    \end{subfigure}
    \hfill
    \begin{subfigure}[b]{0.23\textwidth}
        \centering
        \scalebox{0.45}{
            \begin{othelloboard}{1}
                \posannotation{a1}{10} \posannotation{b1}{11} \posannotation{a2}{8} \posannotation{b2}{9}
                \posannotation{c3}{7} \posannotation{d4}{6} \posannotation{e5}{5} \posannotation{f6}{4}
                \posannotation{g7}{2} \posannotation{h7}{3} \posannotation{g8}{0} \posannotation{h8}{1}
            \end{othelloboard}
        }
        \caption{$T_6$ : Horiz \\ \fontsize{5}{6}\selectfont \texttt{C0C0201008040303}}
    \end{subfigure}
    \hfill
    \begin{subfigure}[b]{0.23\textwidth}
        \centering
        \scalebox{0.45}{
            \begin{othelloboard}{1}
                \posannotation{g1}{8} \posannotation{h1}{10} \posannotation{g2}{9} \posannotation{h2}{11}
                \posannotation{f3}{7} \posannotation{e4}{6} \posannotation{d5}{5} \posannotation{c6}{4}
                \posannotation{a7}{0} \posannotation{b7}{2} \posannotation{a8}{1} \posannotation{b8}{3}
            \end{othelloboard}
        }
        \caption{$T_7$ : DiagA \\ \fontsize{5}{6}\selectfont \texttt{030304081020C0C0}}
    \end{subfigure}
    
    \caption{Istanze e indici di permutazione del pattern BARBEL.}
  \end{figure}
  
\newpage
\subsection{Pattern DIAG3}

\begin{lstlisting}[language=bash, basicstyle=\ttfamily\tiny]
[Pattern: name = DIAG3, mask = 0x0000000000010204]
  [n_squares = 3, n_configurations = 27, n_instances = 4, n_stabilizer = 2]
  Cells:               [C1, B2, A3]
  Transformed masks:   
    0x0000000000010204, 0x0000000000804020, 0x2040800000000000, 0x0402010000000000, 
    0x0000000000804020, 0x2040800000000000, 0x0402010000000000, 0x0000000000010204
  Mask indexes:        [0, 1, 2, 3, 1, 2, 3, 0]
  Unique masks:        
    0x0000000000010204, 0x0000000000804020, 0x2040800000000000, 0x0402010000000000
  Unique mask indexes: [0, 1, 2, 3]
  Transf. functions:   [Id, Rot90, Rot180, Rot270]
  Anti-transf. f.:     [Id, Rot270, Rot180, Rot90]
  Symmetry functions:  [DiagP]
\end{lstlisting}

\begin{figure}[htbp]
    \centering
    \begin{subfigure}[b]{0.23\textwidth}
        \centering
        \scalebox{0.45}{
            \begin{othelloboard}{1}
                \posannotation{c1}{0} \posannotation{b2}{1} \posannotation{a3}{2}
            \end{othelloboard}
        }
        \caption{$T_0$ : Id \\ \fontsize{5}{6}\selectfont \texttt{0000000000010204}}
    \end{subfigure}
    \hfill
    \begin{subfigure}[b]{0.23\textwidth}
        \centering
        \scalebox{0.45}{
            \begin{othelloboard}{1}
                \posannotation{f1}{2} \posannotation{g2}{1} \posannotation{h3}{0}
            \end{othelloboard}
        }
        \caption{$T_1$ : Rot90 \\ \fontsize{5}{6}\selectfont \texttt{0000000000804020}}
    \end{subfigure}
    \hfill
    \begin{subfigure}[b]{0.23\textwidth}
        \centering
        \scalebox{0.45}{
            \begin{othelloboard}{1}
                \posannotation{h6}{2} \posannotation{g7}{1} \posannotation{f8}{0}
            \end{othelloboard}
        }
        \caption{$T_2$ : Rot180 \\ \fontsize{5}{6}\selectfont \texttt{2040800000000000}}
    \end{subfigure}
    \hfill
    \begin{subfigure}[b]{0.23\textwidth}
        \centering
        \scalebox{0.45}{
            \begin{othelloboard}{1}
                \posannotation{a6}{0} \posannotation{b7}{1} \posannotation{c8}{2}
            \end{othelloboard}
        }
        \caption{$T_3$ : Rot270 \\ \fontsize{5}{6}\selectfont \texttt{0402010000000000}}
    \end{subfigure}

    \vspace{0.3cm}

    \begin{subfigure}[b]{0.23\textwidth}
        \centering
        \scalebox{0.45}{
            \begin{othelloboard}{1}
                \posannotation{f1}{0} \posannotation{g2}{1} \posannotation{h3}{2}
            \end{othelloboard}
        }
        \caption{$T_4$ : Vert \\ \fontsize{5}{6}\selectfont \texttt{0000000000804020}}
    \end{subfigure}
    \hfill
    \begin{subfigure}[b]{0.23\textwidth}
        \centering
        \scalebox{0.45}{
            \begin{othelloboard}{1}
                \posannotation{h6}{0} \posannotation{g7}{1} \posannotation{f8}{2}
            \end{othelloboard}
        }
        \caption{$T_5$ : DiagP \\ \fontsize{5}{6}\selectfont \texttt{2040800000000000}}
    \end{subfigure}
    \hfill
    \begin{subfigure}[b]{0.23\textwidth}
        \centering
        \scalebox{0.45}{
            \begin{othelloboard}{1}
                \posannotation{a6}{2} \posannotation{b7}{1} \posannotation{c8}{0}
            \end{othelloboard}
        }
        \caption{$T_6$ : Horiz \\ \fontsize{5}{6}\selectfont \texttt{0402010000000000}}
    \end{subfigure}
    \hfill
    \begin{subfigure}[b]{0.23\textwidth}
        \centering
        \scalebox{0.45}{
            \begin{othelloboard}{1}
                \posannotation{c1}{2} \posannotation{b2}{1} \posannotation{a3}{0}
            \end{othelloboard}
        }
        \caption{$T_6$ : DiagA \\ \fontsize{5}{6}\selectfont \texttt{0402010000000000}}
    \end{subfigure}
    
    \caption{Istanze e indici di permutazione del pattern DIAG3.}
\end{figure}


\end{document}

