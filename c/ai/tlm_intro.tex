%
%  tlm_intro.tex
%
%  Copyright (c) 2026 Roberto Corradini. All rights reserved.
%
%  This file is part of the reversi program
%  http://github.com/rcrr/reversi
%
%  This program is free software; you can redistribute it and/or modify it
%  under the terms of the GNU General Public License as published by the
%  Free Software Foundation; either version 3, or (at your option) any
%  later version.
%
%  This program is distributed in the hope that it will be useful,
%  but WITHOUT ANY WARRANTY; without even the implied warranty of
%  MERCHANTABILITY or FITNESS FOR A PARTICULAR PURPOSE. See the
%  GNU General Public License for more details.
%
%  You should have received a copy of the GNU General Public License
%  along with this program; if not, write to the Free Software
%  Foundation, Inc., 59 Temple Place - Suite 330, Boston, MA  02111-1307, USA
%  or visit the site <http://www.gnu.org/licenses/>.
%


\section{Introduzione e Ambiente di Sviluppo}
Questo documento è la guida per descrivere le basi del modello di Machine-Learning adottato dal
programma Reversi. Il programma è vasto ed articolato, quì si cerca di sintetizzare al meglio le informazioni
basilari per ragionare sullo sviluppo di un modello più performante della valutazione del risultato del gioco,
più sinteticamente chiamata ``evaluation\_function'' o ``ef''.
\begin{itemize}
\item Per cominciare a lavorare avviare \texttt{EMACS} e lanciare una shell con il comando \texttt{``M-x shell''}, quindi posizionarsi
  nella cartella \texttt{\$REVERSI\_HOME/c} e rigenerare il file PDF con il comando \texttt{``make latex-ai''}.
  Per proseguire nella documentazione del modello aprire ed editare il presente documento \texttt{\$REVERSI\_HOME/c/ai/two-model-layer.tex}.
\item Sempre su \texttt{EMACS} aprire una seconda shell con il comando \texttt{``C-u M-x shell''}e posizionarsi nella cartella \texttt{\$REVERSI\_HOME/c}.
  Qui attivare l'ambiente python con il comando \texttt{``source py/.reversi\_venv/bin/activate''} e di seguito
  lanciare gli unit test con il comando \texttt{``PYTHONPATH="./py" python3 -m unittest test.test\_domain''}.
\item Aprire il file \texttt{\$REVERSI\_HOME/c/py/twoml/domain.py} per lavorare sulla logica applicativa, e
  il file\linebreak \texttt{\$REVERSI\_HOME/c/py/test/test\_domain.py} per aggiornare gli unit test.
\item Aprire quindi una terza riga di comando, sempre in \texttt{EMACS} ma di tipo \texttt{term}, con il comando \texttt{``C-u M-x ansi-term''}, e quì eseguire il comado
  \texttt{``gemini''} per avviare il collaboratore sintetico, che in questo contesto chiameremo \texttt{Nia}, diminutivo di Urania, la musa dell'astronomia e della geometria.
  Con il comando \texttt{``@ai/two-model-layer.tex''} fornire a Nia il contesto per poi proseguire nei ragionamenti e nello sviluppo
  del Modello e del Software che lo implementa.
\end{itemize}
