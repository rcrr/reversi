%
%  two-model-layer.tex
%
%  Copyright (c) 2025, 2026 Roberto Corradini. All rights reserved.
%
%  This file is part of the reversi program
%  http://github.com/rcrr/reversi
%
%  This program is free software; you can redistribute it and/or modify it
%  under the terms of the GNU General Public License as published by the
%  Free Software Foundation; either version 3, or (at your option) any
%  later version.
%
%  This program is distributed in the hope that it will be useful,
%  but WITHOUT ANY WARRANTY; without even the implied warranty of
%  MERCHANTABILITY or FITNESS FOR A PARTICULAR PURPOSE. See the
%  GNU General Public License for more details.
%
%  You should have received a copy of the GNU General Public License
%  along with this program; if not, write to the Free Software
%  Foundation, Inc., 59 Temple Place - Suite 330, Boston, MA  02111-1307, USA
%  or visit the site <http://www.gnu.org/licenses/>.
%

%
% The most part of the Reversi source code and documentation is written in English.
% This file started with the specific intention to experiment new evolutions for the
% program and for the design of the Machine-Learning Model dealing with the help and the
% interaction with AI agents.
% It has become evident that the language of choice is not that relevant for the AI, and
% it is so easy to get a quality translation from AI today, that you are not minding me
% to choose Italian here for the seek of my confort instead of English.
%

%
% TO-DO
% - Fare review del documento e portarlo in pari con gli sviluppi python del modulo domain.
% - Manca il codice che esegue la generazione del pattern.
% - Documentare tutta la evaluation_function.
% - Codice per la creazione del modello e per il calcolo delle prestazioni.
% - Definizione del nuovo modello RNNM (Reversi Neural Network Model)
%

\documentclass{article}
\renewcommand{\familydefault}{\sfdefault}
\usepackage{sfmath}

\usepackage{amsmath}
\usepackage{othelloboard}
\usepackage{subcaption}
\usepackage{graphicx}

\usepackage[a4paper, left=2cm, right=2cm, top=3cm, bottom=3cm]{geometry}

\usepackage{listings}
\usepackage[dvipsnames]{xcolor}
\definecolor{mauve}{rgb}{0.88, 0.69, 1.0}
\usepackage{fancyvrb}
\usepackage{enumitem}

\usepackage{textcomp}
\usepackage[T1]{fontenc}
\usepackage{url}

\usepackage{bm}
\usepackage[utf8]{inputenc}

\usepackage{xcolor}
\usepackage{hyperref}

\hypersetup{
    colorlinks=true,      % Rimuove i riquadri e colora il testo
    linkcolor=blue,       % Colore per i riferimenti interni (sezioni, figure, equazioni)
    citecolor=blue,       % Colore per le citazioni bibliografiche
    filecolor=magenta,    
    urlcolor=cyan,
    pdftitle={Pattern Symmetries in Reversi},
}

\usepackage{amssymb}

\lstset{
  language=SQL,
  basicstyle=\ttfamily\tiny,
  keywordstyle=\color{blue},
  stringstyle=\color{mauve},
  commentstyle=\color{gray},
  numbers=none,
  breaklines=true,
  showstringspaces=false,
  frame=single
}

\lstset{
  language=bash,
  basicstyle=\ttfamily\tiny,
  keywordstyle=\color{blue},
  commentstyle=\color{gray},
  stringstyle=\color{mauve},
  breaklines=true,
  frame=single,
  showstringspaces=false
}

\lstset{
  language=C,
  basicstyle=\ttfamily\tiny,
  keywordstyle=\color{blue},
  stringstyle=\color{mauve},
  commentstyle=\color{gray},
  numbers=none,
  breaklines=true,
  showstringspaces=false,
  frame=single,
  literate={<<}{{$\ll$}}2 {>>}{{$\gg$}}2
}

% 
% M-:
% (setq preview-scale-function (lambda () 1.5))
%

%
% Per scrivere le vocali accentate in Italiano:
%
% $ setxkbmap -layout us -variant intl
%

\title{Reversi}
\author{Roberto Corradini \\[0.2cm]
\small \texttt{rob\_corradini@yahoo.com} \\[0.1cm]
\url{https://github.com/rcrr/reversi}}
\date{\today}

\begin{document}
\maketitle
\vspace{1cm}
\begin{abstract}
\normalsize
Reversi è un ecosistema di programmi e strumenti informatici per studiare il gioco da tavolo Othello, anche chiamato Reversi.
L'intento è studiare, capire, sviluppare algoritmi e modelli che possano progredire nella realizzazione di un agente
sintetico ed autonomo che sia in grado di giocare una partita in maniera autorevole. E in prospettiva di tendere alla
risoluzione matematica del gioco, o quantomeno ad una sua approssimazione probabilistica.
I programmi sono scritti principalmente in ANSI C, Python, SQL, con altri linguaggi come Common Lisp, Java, Clojure, R usati
per sperimentazioni, ma non rilevanti nella sostanza.
Il corpo principale del programma è sviluppato in C, e comprende l'endgame solver, il generatore di partite random, il
solutore integrato con il database REGAB, e le funzioni di risoluzione del modello di ML basato sulla Regressione Logistica.
Il Database SQL REGAB, sviluppato in PostgreSQL, ed in linguaggio procedurale PL/pgSQL, contiene le posizioni random
che poi vengono risolte e taggate con il valore vero del game.
Il codice in C è stato poi portato in Python, per poter operare più agilmente con l'ecosistema attivo del Machine-Learning, AI,
calcolo scientifico, acceleratori GPU, e tutto quanto serve per la ricerca di frontiera.
Il codice in C rimane più performante per la risoluzione delle posizioni, il codice in Python è più consono allo sviluppo
dei Modelli per la definizione di evaluation function più evolute.
\end{abstract}

\newpage

\section{Introduzione e Ambiente di Sviluppo}
Questo documento è la guida per descrivere le basi del modello di Machine-Learning adottato dal
programma Reversi. Il programma è vasto ed articolato, quì si cerca di sintetizzare al meglio le informazioni
basilari per ragionare sullo sviluppo di un modello più performante della valutazione del risultato del gioco,
più sinteticamente chiamata ``evaluation\_function'' o ``ef''.
\begin{itemize}
\item Per cominciare a lavorare avviare \texttt{EMACS} e lanciare una shell con il comando \texttt{``M-x shell''}, quindi posizionarsi
  nella cartella \texttt{\$REVERSI\_HOME/c} e rigenerare il file PDF con il comando \texttt{``make latex-ai''}.
  Per proseguire nella documentazione del modello aprire ed editare il presente documento \texttt{\$REVERSI\_HOME/c/ai/two-model-layer.tex}.
\item Sempre su \texttt{EMACS} aprire una seconda shell con il comando \texttt{``C-u M-x shell''}e posizionarsi nella cartella \texttt{\$REVERSI\_HOME/c}.
  Qui attivare l'ambiente python con il comando \texttt{``source py/.reversi\_venv/bin/activate''} e di seguito
  lanciare gli unit test con il comando \texttt{``PYTHONPATH="./py" python3 -m unittest test.test\_domain''}.
\item Aprire il file \texttt{\$REVERSI\_HOME/c/py/twoml/domain.py} per lavorare sulla logica applicativa, e
  il file\linebreak \texttt{\$REVERSI\_HOME/c/py/test/test\_domain.py} per aggiornare gli unit test.
\item Aprire quindi una terza riga di comando, sempre in \texttt{EMACS} ma di tipo \texttt{term}, con il comando \texttt{``C-u M-x ansi-term''}, e quì eseguire il comado
  \texttt{``gemini''} per avviare il collaboratore sintetico, che in questo contesto chiameremo \texttt{Nia}, diminutivo di Urania, la musa dell'astronomia e della geometria.
  Con il comando \texttt{``@ai/two-model-layer.tex''} fornire a Nia il contesto per poi proseguire nei ragionamenti e nello sviluppo
  del Modello e del Software che lo implementa.
\end{itemize}

\section{Rappresentazione del Dominio}

\subsection{La scacchiera (Board)}

La BOARD Reversi è composta da 64 celle, otto righe ed otto colonne, numerate da 0, in alto a sinistra (North-West, NW),
a 63, in basso a destra (South-East, SE).

La configurazione della BOARD è rappresentata con due interi a 64 bit (indifferentemente senza segno \texttt{uint64} o con segno \texttt{int64}).
Ognuno dei 64 bit è associato ad una cella. Il bit meno significativo è associato ad \texttt{a1}, quello più significativo ad \texttt{h8},
procedendo da sinistra verso destra per righe.

Il primo intero rappresenta le celle del \texttt{GIOCATORE (PLAYER)} che deve effetuare la mossa, il secondo le celle dell'\texttt{AVVERSARIO (OPPONENT)}.

Con due valori \texttt{int64} viene quindi rappresentata la \texttt{POSIZIONE (GAME\_POSITION)}, come mostrato in figura~\ref{fig:reversi-bitboard-patterns}.

Unico vincolo, o invariante, della rappresentazione è la formula: \texttt{PLAYER \&\& OPPONENT == 0},
che formalizza il divieto di avere sia un disco del giocatore che uno dell'avversario contemporaneamente nella stessa cella.

Quando una cella ha zero sia per il giocatore che per l'avversario è vuota.


\begin{figure}[htbp]
  \centering
  \begin{subfigure}[b]{0.45\textwidth}
    \centering
    \scalebox{0.7}{
      \begin{othelloboard}{1}
        \dotmarkings
        \othelloarrayfirstrow   {0}{0}{0}{0}{0}{0}{0}{0}
        \othelloarraysecondrow  {0}{0}{0}{0}{0}{0}{0}{0}
        \othelloarraythirdrow   {0}{0}{0}{0}{0}{0}{0}{0}
        \othelloarrayfourthrow  {0}{0}{0}{1}{2}{0}{0}{0}
        \othelloarrayfifthrow   {0}{0}{0}{2}{1}{0}{0}{0}
        \othelloarraysixthrow   {0}{0}{0}{0}{0}{0}{0}{0}
        \othelloarrayseventhrow {0}{0}{0}{0}{0}{0}{0}{0}
        \othelloarrayeighthrow  {0}{0}{0}{0}{0}{0}{0}{0}
      \end{othelloboard}
    }
    \caption{Initial game position.}
  \end{subfigure}
  \hfill
  \begin{subfigure}[b]{0.45\textwidth}
    \centering
    \scalebox{0.7}{
      \begin{othelloboard}{1}
        \dotmarkings
        \othelloarrayfirstrow   {0}{0}{0}{0}{0}{0}{0}{0}
        \othelloarraysecondrow  {0}{0}{0}{0}{0}{0}{0}{0}
        \othelloarraythirdrow   {0}{0}{0}{0}{0}{0}{0}{0}
        \othelloarrayfourthrow  {0}{0}{0}{0}{0}{0}{0}{0}
        \othelloarrayfifthrow   {0}{0}{0}{0}{0}{0}{0}{0}
        \othelloarraysixthrow   {0}{0}{0}{0}{0}{0}{0}{0}
        \othelloarrayseventhrow {0}{0}{0}{0}{0}{0}{0}{0}
        \othelloarrayeighthrow  {0}{0}{0}{0}{0}{0}{0}{0}
        %
        \posannotation{a1}{\small 00}
        \posannotation{b1}{\small 01}
        \posannotation{c1}{\small 02}
        \posannotation{d1}{\small ...}
        \posannotation{a2}{\small 08}
        \posannotation{b2}{\small 09}
        \posannotation{c2}{\small ...}
        \posannotation{g8}{\small ...}
        \posannotation{h8}{\small 63}
      \end{othelloboard}
    }
    \caption{Rappresentazione della bitboard.}
  \end{subfigure}

  \vskip\baselineskip
  
  \begin{subfigure}[b]{0.45\textwidth}
    \centering
    \scalebox{0.7}{
      \begin{othelloboard}{1}
        \dotmarkings
        \othelloarrayfirstrow   {0}{0}{0}{0}{0}{0}{0}{0}
        \othelloarraysecondrow  {0}{0}{0}{0}{0}{0}{0}{0}
        \othelloarraythirdrow   {0}{0}{0}{0}{0}{0}{0}{0}
        \othelloarrayfourthrow  {0}{0}{0}{0}{0}{0}{0}{0}
        \othelloarrayfifthrow   {0}{0}{0}{0}{0}{0}{0}{0}
        \othelloarraysixthrow   {0}{0}{0}{0}{0}{0}{0}{0}
        \othelloarrayseventhrow {0}{0}{0}{0}{0}{0}{0}{0}
        \othelloarrayeighthrow  {0}{0}{0}{0}{0}{0}{0}{0}
        %
        \posannotation{a1}{\small 00}
        \posannotation{b1}{\small 01}
        \posannotation{c1}{\small 02}
        \posannotation{d1}{\small 03}
        \posannotation{e1}{\small 04}
        \posannotation{f1}{\small 05}
        \posannotation{g1}{\small 06}
        \posannotation{h1}{\small 07}
      \end{othelloboard}
    }
    \caption{EDGE pattern.}
  \end{subfigure}
  \hfill
  \begin{subfigure}[b]{0.45\textwidth}
    \centering
    \scalebox{0.7}{
      \begin{othelloboard}{1}
        \dotmarkings
        \othelloarrayfirstrow   {0}{0}{0}{0}{0}{0}{0}{0}
        \othelloarraysecondrow  {0}{0}{0}{0}{0}{0}{0}{0}
        \othelloarraythirdrow   {0}{0}{0}{0}{0}{0}{0}{0}
        \othelloarrayfourthrow  {0}{0}{0}{0}{0}{0}{0}{0}
        \othelloarrayfifthrow   {0}{0}{0}{0}{0}{0}{0}{0}
        \othelloarraysixthrow   {0}{0}{0}{0}{0}{0}{0}{0}
        \othelloarrayseventhrow {0}{0}{0}{0}{0}{0}{0}{0}
        \othelloarrayeighthrow  {0}{0}{0}{0}{0}{0}{0}{0}
        %
        \posannotation{a2}{\small 00}
        \posannotation{b2}{\small 01}
        \posannotation{c2}{\small 02}
        \posannotation{d2}{\small 03}
        \posannotation{e2}{\small 04}
        \posannotation{f2}{\small 05}
        \posannotation{g2}{\small 06}
        \posannotation{h2}{\small 07}
      \end{othelloboard}
    }
    \caption{R2 pattern.}
  \end{subfigure}

  \caption{Ordine dei bit per la definizione della bitboard e dei pattern.}
  \label{fig:reversi-bitboard-patterns}
\end{figure}

\subsection{Le posizioni (Game Positions)}

Ogni posizione è descritta dal \texttt{VALORE} dell'end-game ottenuto risolvendo la posizione stessa in maniera esatta con
apposito programma che applica l'algoritmo \texttt{MINIMAX} (o sue varianti sempre esatte ma più efficienti come \texttt{ALPHA-BETA}).

I \texttt{VALORI} possibili di una posizione sono i numeri interi pari compresi nell'intervallo \texttt{[-64..+64]}, pari alla differenza
di dischi dei due colori risultante alla fine della partita.
Zero significa pareggio, valori negativi una sconfitta, valori positivi una vittoria.

Le posizioni random, generate per costituire il dataset per il machine-learning (ML), sono archiviate in un database SQL chiamato REGAB (Reversi End-GAme Base).
Prendiamo una posizione risolta da REGAB con il comando SQL:

\begin{lstlisting}[language=SQL]
  SELECT seq, mover, opponent, empty_count, game_value, best_move FROM regab_prng_gp WHERE batch_id = 7 AND empty_count = 20 AND status = 'CMS' LIMIT 1;
\end{lstlisting}

{\tiny
\begin{verbatim}
   seq    |        mover        |       opponent       | empty_count | game_value | best_move 
----------+---------------------+----------------------+-------------+------------+-----------
 68230056 | 4611717676283199524 | -7855295674223658936 |          20 |         10 | F8
(1 row)
\end{verbatim}
}

Possiamo poi ricavare dal database ulteriori informazioni, come una vista 2D della posizione:
  
\begin{lstlisting}[language=SQL]
  SELECT game_position_pp_mop(mover, opponent, player) AS pos_68230056 FROM regab_prng_gp WHERE seq = 68230056;
\end{lstlisting}

{\tiny
\begin{verbatim}
     pos_68230056      
-----------------------
    a b c d e f g h   +
 1  . . @ O . @ O .   +
 2  . . . O O O O .   +
 3  . . O O O O @ .   +
 4  @ @ O @ O @ @ @   +
 5  . @ O @ O O @ @   +
 6  . O @ @ @ O O .   +
 7  . . O O O O O O   +
 8  . O . . O . @ O   +
 Player to move: BLACK
(1 row)
\end{verbatim}
}

Dove il carattere \texttt{'.'} significa cella vuota, \texttt{'@'} disco nero, e \texttt{'O'} disco bianco.
In questo caso il ``player to move'' è il \texttt{BLACK} che quindi è il \texttt{PLAYER}, mentre il giocatore bianco
è l' \texttt{OPPONENT}.

O una rappresentazione come stringa che poi possiamo usare come input per l'eseguibile che calcola il valore del game:

\begin{lstlisting}[language=SQL]
  SELECT game_position_to_string((mover, opponent, player)) AS pos_as_string_68230056 FROM regab_prng_gp WHERE seq = 68230056;
\end{lstlisting}

{\tiny
\begin{verbatim}
                      pos_as_string_68230056                       
-------------------------------------------------------------------
 ..bw.bw....wwww...wwwwb.bbwbwbbb.bwbwwbb.wbbbww...wwwwww.w..w.bwb
(1 row)
\end{verbatim}
}

Dove il carattere \texttt{'.'} significa cella vuota, \texttt{'b'} disco nero, \texttt{'w'} disco bianco. Il carattere più a sinistra
rappresenta la cella \texttt{00} (a1), muovendosi poi verso destra fino al carattere 64 che rappresenta la cella \texttt{63} (h8);
l'ultimo carattere a destra rappresenta invece il giocatore che deve muovere, \texttt{'b'} per \texttt{BLACK} e \texttt{'w'} per \texttt{WHITE}.

Eseguiamo il programma \texttt{endgame\_solver} dove il flag \texttt{-s es} richiama l' ``exact solver'':

\begin{lstlisting}[language=bash]
  ./build/bin/endgame_solver -s es -g ..bw.bw....wwww...wwwwb.bbwbwbbb.bwbwwbb.wbbbww...wwwwww.w..w.bwb
\end{lstlisting}

{\tiny
\begin{verbatim}
    a b c d e f g h 
 1  . . @ O . @ O . 
 2  . . . O O O O . 
 3  . . O O O O @ . 
 4  @ @ O @ O @ @ @ 
 5  . @ O @ O O @ @ 
 6  . O @ @ @ O O . 
 7  . . O O O O O O 
 8  . O . . O . @ O 
Player to move: BLACK

[node_count=45798972, leaf_count=8265202]
Final outcome: best move=F8, position value=10
\end{verbatim}
}

Il risultato è la posizione data da risolvere con il conto dei nodi e delle foglie del ``game-tree'' navigati dall'algoritmo.
Il programma, lanciato in batch mode, ha risolto e quindi taggato le posizioni del database REGAB con il valore esatto del game.

Ai fini del machine learning (ML) un record posizione di esempio con header è rappresentato come segue:

{\tiny
\begin{verbatim}
             PLAYER ;             OPPONENT ; GAME_VALUE
4611717676283199524 ; -7855295674223658936 ;         10
\end{verbatim}
}

Dove:

\begin{description}[leftmargin=*,widest=GAME\_VALUE,labelsep=1em]
\item[PLAYER] è la configurazione della board dei dischi occupati dal giocatore che deve muovere, in formato \texttt{int64} con segno.
  Il bit meno significativo è la casella \texttt{a1}, si procede per righe, come specificato sopra.
\item[OPPONENT] è la configurazione della board dei dischi occupati dall'avversario, stesso formato usato per \texttt{PLAYER}.
\item[GAME\_VALUE] è il valore esatto del game
\end{description}

Il valore del \texttt{GAME\_VALUE} è ottenuto con un software di soluzione esatta della posizione.
Il database contiene decine di milioni di record posizione risolti.
Le posizioni sono generate eseguendo delle mosse legali scelte a random a partire dalle condizione iniziali.

\section{Definizioni in Python degli oggetti Square, Move, SquareSet e Board}
Il modulo \texttt{domain}, implementato nel file \texttt{\$REVERSI\_HOME/c/py/twoml/domain.py}, definisce gli oggetti di base del dominio applicativo:
Square, Move, SquareSet e Board.

La classe Square è definita come un sottotipo di numpy.uint8 e definisce pochi metodi di base per convertire stringhe del tipo \texttt{F7} in numeri
interi, 53 nel caso della casella F7, e vice versa.

La classe Move è derivata da Square, aggiungendo tre valori utili alla definizione della dinamica del gioco ed ai valori trovati nel database.
Il valore \texttt{PA} indica la mossa di passare all'avversario quando non ci sono mosse legali disponibili, quello \texttt{NA} per non disponibile, è
usato dal programma quando la mossa non è ancora stata definita, mentre \texttt{UN} è usato quando il valore è indefinito.

La classe SquareSet è definita come sottotipo di numpy.uint64 e definisce il set delle 64 caselle della board, è il componente di base poi per definire
la classe Board, che è infatti costruita su due attributi, mover ed opponent, definiti come SquareSet.
Un oggetto di tipo SquareSet può essere costruito da qualsiasi rappresentazione di un intero a 64 bit, e può poi essere convertito in queste all'occorrenza
con una serie di metodi definiti dalla classe. Altri metodi accessori sono \texttt{bsr}, per calcolare la posizione della ultima casella del set,
\texttt{to\_square\_list} per avere una lista delle Square del set, \texttt{to\_square\_array} per un array, o \texttt{count} per la conta delle caselle del set.
La classe inoltre definisce le trasformazioni a cui una board può essere soggetta grazie alle sue molteplici simmetrie, queste trasformazioni sono il focus
del prossimo paragrafo.

La classe Board è l'implementazione in Reversi del concetto di bitboard, consolida lo stato del game nei suoi due attributi chiamati mover ed opponent,
ed implementa le regole e la meccanica del gioco. Oltre ai classici metodi di utilità di base quali \texttt{clone} e \texttt{print}, vi sono le funzioni
caratteristiche quali \texttt{legal\_moves} e \texttt{make\_move}, ed altre accessorie del tipo \texttt{has\_to\_pass}, \texttt{empties}, or \texttt{is\_move\_legal}.
In aggiunta vi sono poi tutte le trasformazioni definite per gli SquareSet, riportati sulla Board.
Le funzioni \texttt{legal\_moves}, \texttt{make\_move} e la accessoria \texttt{flips} sono implementate tramite due funzioni interne,
\texttt{\_kogge\_stone\_lms} e \texttt{\_kogge\_stone\_mm} che sono basate sulla struttura interna della bitboard. Nella versione implementata in Python non tutto
il parallelismo potenziale delle istruzioni vettoriali è sfruttatto, cosa che invece avviene pienamente nella implementazione in ANSI C.

\section{Geometria della Scacchiera: Il Gruppo Diedrale \texorpdfstring{$\bm{D_4}$}{D4}}

\subsection{Rotazioni della bitboard}

Sia gli square set che le board sono passibili di otto differenti trasformazioni, che ruotano rigidamente, oppure specchiano la bitboard rispetto gli assi di simmetria
naturali. Le trasformazioni sono poi usate per calcolare gli indici delle istanze dei diversi pattern.
La board può essere ruotata di zero, 90, 180, 270 gradi in senso orario, e può essere specchiata secondo gli assi verticale, orizontale, diagonale NW-SE e diagonale NE-SW.
Sono poi definite le operazioni delle otto antitrasformazioni che riportano la bitboard nella condizione originale.

A titolo di esempio la board:

{\tiny
\begin{verbatim}
A1 B1 C1 D1 E1 F1 G1 H1
A2 B2 C2 D2 E2 F2 G2 H2
A3 B3 C3 D3 E3 F3 G3 H3
A4 B4 C4 D4 E4 F4 G4 H4
A5 B5 C5 D5 E5 F5 G5 H5
A6 B6 C6 D6 E6 F6 G6 H6
A7 B7 C7 D7 E7 F7 G7 H7
A8 B8 C8 D8 E8 F8 G8 H8
\end{verbatim}
}

ruotata in senso orario di 90 gradi diventa:

{\tiny
\begin{verbatim}
A8 A7 A6 A5 A4 A3 A2 A1
B8 B7 B6 B5 B4 B3 B2 B1
C8 C7 C6 C5 C4 C3 C2 C1
D8 D7 D6 D5 D4 D3 D2 D1
E8 E7 E6 E5 E4 E3 E2 E1
F8 F7 F6 F5 F4 F3 F2 F1
G8 G7 G6 G5 G4 G3 G2 G1
H8 H7 H6 H5 H4 H3 H2 H1
\end{verbatim}
}

Le otto trasformazioni sono quindi definite dai metodi omonimi nelle due classi SquareSet e Board come
descritti nella tabella \ref{tab:bitboard_transformations}.
Inoltre le due classi hanno definiti i metodi \texttt{transformations} ed \texttt{anti\_transformations} che ritornano
un array di otto posizioni contenente le bitboard trasformate o anti-trasformate.

\begin{table}[h]
\centering
\caption{Trasformazioni della bitboard}
\label{tab:bitboard_transformations}
\begin{tabular}{|c|l|l|l|}
\hline
\textbf{Pos.} & \textbf{Descrizione} & \textbf{Trasformazione} & \textbf{Anti-Trasformazione} \\ \hline
0 & Rotazione zero gradi oraria  & \texttt{trans\_identity}               & \texttt{trans\_identity}               \\ \hline
1 & Rotazione 90 gradi oraria    & \texttt{trans\_rotate\_90c}            & \texttt{trans\_rotate\_90a}            \\ \hline
2 & Rotazione 180 gradi oraria   & \texttt{trans\_rotate\_180}            & \texttt{trans\_rotate\_180}            \\ \hline
3 & Rotazione 270 gradi oraria   & \texttt{trans\_rotate\_90a}            & \texttt{trans\_rotate\_90c}            \\ \hline
4 & Specchiatura asse verticale  & \texttt{trans\_reflection\_vertical}   & \texttt{trans\_reflection\_vertical}   \\ \hline
5 & Specchiatura diagonale H1-A8 & \texttt{trans\_reflection\_diag\_h1a8} & \texttt{trans\_reflection\_diag\_h1a8} \\ \hline
6 & Specchiatura asse orizontale & \texttt{trans\_reflection\_horizontal} & \texttt{trans\_reflection\_horizontal} \\ \hline
7 & Specchiatura diagonale A1-H8 & \texttt{trans\_reflection\_diag\_a1h8} & \texttt{trans\_reflection\_diag\_a1h8} \\ \hline
\end{tabular}
\end{table}

\subsection{Definizione e Composizione del Gruppo}
\label{sec:gruppo_diedrale}

Le trasformazioni geometriche che agiscono sulla scacchiera del Reversi formano un gruppo matematico
noto come \textbf{Gruppo Diedrale di ordine 8}, indicato con il simbolo $\bm{D_4}$.
Questo gruppo rappresenta l'insieme completo delle simmetrie di un quadrato.

\subsection{Composizione del Gruppo}
Il gruppo $G = \{T_0, T_1, \dots, T_7\}$ è composto da 8 elementi fondamentali:
\begin{itemize}
    \item \textbf{4 Rotazioni:} $0^\circ$ (l'elemento \textit{Identità} $e$), $90^\circ$, $180^\circ$ e $270^\circ$.
    \item \textbf{4 Riflessioni:} Orizzontale, Verticale e le due Diagonali.
\end{itemize}

\begin{table}[h]
\centering
\caption{Elementi del Gruppo Diedrale $D_4$ e relative Anti-trasformazioni.}
\label{tab:d4_inverses}
\begin{tabular}{@{}clllc@{}}
\hline
\textbf{Indice} & \textbf{Operatore} & \textbf{Descrizione Geometrica} & \textbf{Inverso ($T^{-1}$)} & \textbf{Ordine} \\ \hline
0 & $e$      & Identità ($0^\circ$)          & $T_0$ & 1 \\
1 & $r$      & Rotazione $90^\circ$ (CW)     & $T_3$ & 4 \\
2 & $r^2$    & Rotazione $180^\circ$         & $T_2$ & 2 \\
3 & $r^3$    & Rotazione $270^\circ$ (CW)    & $T_1$ & 4 \\
4 & $s$      & Riflessione Orizzontale       & $T_4$ & 2 \\
5 & $sr$     & Riflessione Diagonale (P)     & $T_5$ & 2 \\
6 & $sr^2$   & Riflessione Verticale         & $T_6$ & 2 \\
7 & $sr^3$   & Riflessione Diagonale (A)     & $T_7$ & 2 \\ \hline
\end{tabular}
\end{table}

L'ordine di un elemento \(T\in D_{4}\) rappresenta il ciclo di ripetizione della trasformazione geometrica.
Gli elementi di ordine 2, detti involuzioni, sono auto-inversi (\(T=T^{-1}\)), semplificando la logica di mappatura tra istanze simmetriche.

\subsection{Struttura Algebrica e Generatori}

Il gruppo diedrale $D_4$ può essere descritto in modo compatto attraverso i suoi \textbf{generatori}. Invece di considerare le otto trasformazioni come elementi isolati, le definiamo come combinazioni di due operazioni fondamentali.

\subsection{I Generatori: Rotazione e Riflessione}
Il gruppo $D_4$ è generato da due elementi fondamentali:
\begin{itemize}
    \item $r$: una rotazione di $90^\circ$ in senso orario (ordine 4, per cui $r^4 = e$).
    \item $s$: una riflessione rispetto all'asse orizzontale (ordine 2, per cui $s^2 = e$).
\end{itemize}

\subsection{Relazioni Fondamentali e Proprietà Non-Abeliana}
La struttura del gruppo è determinata dalla relazione di interazione tra questi due generatori. In $D_4$, vale la seguente relazione fondamentale:
\begin{equation}
    rs = sr^{-1} \quad (\text{ovvero } rs = sr^3)
\end{equation}
Questa uguaglianza dimostra formalmente che il gruppo \textbf{non è abeliano}: applicare una rotazione seguita da una riflessione ($rs$) non equivale ad applicare una riflessione seguita dalla stessa rotazione ($sr$). 

\subsection{Scomposizione degli Elementi}
Grazie ai generatori $r$ e $s$, l'insieme degli otto elementi di $D_4$ può essere ripartito in due sottoinsiemi disgiunti:
\begin{enumerate}
    \item \textbf{Sottogruppo delle Rotazioni (Ciclico):} 
    \begin{equation}
        R = \{e, r, r^2, r^3\}
    \end{equation}
    Questi elementi mantengono l'orientamento (chirale) originale della scacchiera.
    
    \item \textbf{Insieme delle Riflessioni (Classe Laterale):}
    \begin{equation}
        S = \{s, sr, sr^2, sr^3\}
    \end{equation}
    Ogni elemento di questo insieme rappresenta una riflessione. Algebricamente, una riflessione ``composta'' (come quella diagonale) è vista come l'atto di ribaltare la scacchiera ($s$) e successivamente ruotarla per raggiungere l'asse di simmetria desiderato.
\end{enumerate}

\subsection{Interpretazione Geometrica}
La notazione $sr^k$ spiega la natura delle simmetrie speculari:
\begin{itemize}
    \item $s$: Riflessione Orizzontale.
    \item $sr$: Riflessione rispetto alla Diagonale Principale.
    \item $sr^2$: Riflessione Verticale.
    \item $sr^3$: Riflessione rispetto alla Diagonale Anti-principale.
\end{itemize}
Questa classificazione è di fondamentale importanza per il software: permette di mappare ogni possibile simmetria della scacchiera utilizzando solo due funzioni atomiche (\textit{rotate} e \textit{reflection}) e le loro composizioni.

\subsection{Implementazione Computazionale (Parallel Bit Manipulation)}

Dal punto di vista dell'efficienza algoritmica, le trasformazioni del gruppo $D_4$ non sono implementate come rotazioni geometriche classiche, ma come sequenze di \textit{Parallel Bit Manipulation} (tecnica nota come \textit{Delta Swap}). Questa scelta permette di eseguire ogni simmetria in tempo costante $O(1)$ senza l'ausilio di cicli o tabelle di lookup pesanti.

\subsection{Composizione delle Trasformazioni}
Sebbene algebricamente il gruppo sia generato da rotazioni e riflessioni, nel software è più efficiente utilizzare le riflessioni assiali e diagonali come primitive atomiche. Le rotazioni vengono quindi derivate dalla composizione di queste ultime, come illustrato nella Tabella \ref{tab:code_mapping}.

\begin{table}[h]
\centering
\caption{Mappatura delle funzioni del software sugli elementi di $D_4$.}
\label{tab:code_mapping}
\begin{tabular}{lll}
\hline
\textbf{Elemento Algebrico} & \textbf{Funzione nel Software} & \textbf{Composizione Logica} \\ \hline
$e$ (Identità) & \texttt{trans\_identity} & - \\
$s$ (Riflessione Orizzontale) & \texttt{trans\_reflection\_horizontal} & Primitive \\
$sr^2$ (Riflessione Verticale) & \texttt{trans\_reflection\_vertical} & Primitive \\
$sr$ (Riflessione Diagonale) & \texttt{trans\_reflection\_diag\_h1a8} & Primitive \\
$r$ (Rotazione $90^\circ$ CW) & \texttt{trans\_rotate\_90c} & \texttt{refl\_diag\_h1a8} $\circ$ \texttt{refl\_horiz} \\
$r^2$ (Rotazione $180^\circ$) & \texttt{trans\_rotate\_180} & \texttt{refl\_horiz} $\circ$ \texttt{refl\_vert} \\
$r^3$ (Rotazione $90^\circ$ ACW) & \texttt{trans\_rotate\_90a} & \texttt{refl\_horiz} $\circ$ \texttt{refl\_diag\_h1a8} \\ \hline
\end{tabular}
\end{table}

\subsection{Analisi dell'Efficienza}
L'approccio utilizzato riduce il \textit{branching} della CPU. Operazioni come \texttt{trans\_reflection\_diag\_h1a8} utilizzano maschere costanti (\textit{k1, k2, k4}) e operazioni di \textit{shift} e \textit{XOR} per scambiare interi blocchi di bit simultaneamente. Questo garantisce che la valutazione dei pattern rimanga performante anche durante ricerche profonde nell'albero di gioco.


\section{Teoria dei Pattern e Canonizzazione}

\subsection{Proprietà Algebriche e Simmetrie del Pattern}
Sebbene il sottogruppo delle sole rotazioni sia \textit{Abeliano} (commutativo), l'intero gruppo $D_4$ \textbf{non è Abeliano}. L'ordine con cui vengono applicate le trasformazioni è determinante: ad esempio, ruotare la scacchiera e poi rifletterla produce un risultato differente rispetto a rifletterla e poi ruotarla ($a \cdot b \neq b \cdot a$).

Per la corretta catalogazione dei pattern e la riduzione dei parametri nel modello di Machine Learning, è necessario utilizzare il vocabolario della \textit{Teoria dei Gruppi}:

\begin{enumerate}
    \item \textbf{Azione del Gruppo (Group Action):} Rappresenta l'applicazione di una trasformazione $T \in G$ alla maschera binaria $\mathcal{M}$ di un pattern $\mathcal{P}$.
    \item \textbf{Orbita ($\text{Orb}$):} L'insieme di tutte le configurazioni uniche che il pattern assume sotto l'azione di $G$. Nel software, questo corrisponde alle istanze fisiche distinte sulla scacchiera.
    \item \textbf{Stabilizzatore ($\text{Stab}$):} Il sottogruppo di trasformazioni che lasciano la maschera del pattern bitwise identica ($T(\mathcal{M}) = \mathcal{M}$). Questi elementi definiscono le invarianze geometriche del pattern.
    \item \textbf{Automorfismo:} Una trasformazione appartenente allo stabilizzatore che, pur mantenendo invariata la maschera, permuta l'ordine interno delle celle del pattern.
\end{enumerate}

\subsection{Il Teorema Orbita-Stabilizzatore}
La relazione tra il numero di istanze di un pattern e le sue simmetrie è governata dal \textbf{Teorema Orbita-Stabilizzatore}:
\begin{equation}
    |G| = |\text{Orb}(\mathcal{P})| \times |\text{Stab}(\mathcal{P})|
\end{equation}
Poiché l'ordine del gruppo $|G|$ è esattamente 8, la cardinalità dell'orbita (ovvero il numero di istanze uniche) deve essere necessariamente un divisore di 8. Di conseguenza, un pattern può possedere solo \textbf{1, 2, 4 o 8 istanze} sulla scacchiera.

\subsection{Classificazione Geometrica e Molteplicità dei Pattern}

Il Teorema Orbita-Stabilizzatore non definisce solo il numero di istanze, ma permette di classificare ogni pattern in categorie di simmetria basate sui sottogruppi di $D_4$. Poiché l'ordine del gruppo è $|G|=8$, la cardinalità dell'orbita (le istanze fisiche) è vincolata dalla relazione:
\begin{equation}
    |\text{Orb}(\mathcal{P})| = \frac{8}{|\text{Stab}(\mathcal{P})|}
\end{equation}

In base alla geometria della maschera $\mathcal{M}$, i pattern possono essere raggruppati in quattro classi fondamentali di molteplicità:

\begin{enumerate}
    \item \textbf{Pattern Asimmetrici ($|\text{Orb}|=8$):} Lo stabilizzatore contiene solo l'identità ($|\text{Stab}|=1$). Il pattern non possiede simmetrie interne; ogni trasformazione genera un'istanza distinta. È il caso più comune per pattern irregolari.
    
    \item \textbf{Pattern a Simmetria Singola ($|\text{Orb}|=4$):} Lo stabilizzatore ha ordine 2. Il pattern è invariante rispetto a una sola trasformazione (es. solo riflessione orizzontale, o solo rotazione di $180^\circ$). Esistono 4 istanze distinte sulla scacchiera.
    
    \item \textbf{Pattern a Simmetria Doppia ($|\text{Orb}|=2$):} Lo stabilizzatore ha ordine 4. È il caso dei pattern con simmetria bi-assiale o rotazionale di $90^\circ$ (come il pattern \texttt{DIAG8} analizzato precedentemente).
    
    \item \textbf{Pattern Totalmente Simmetrici ($|\text{Orb}|=1$):} Lo stabilizzatore coincide con l'intero gruppo $D_4$ ($|\text{Stab}|=8$). Il pattern è invariante rispetto a tutte le rotazioni e riflessioni (es. un quadrato centrale $2\times2$ o la croce centrale).
\end{enumerate}

\subsection{Lo Spazio degli Stati e Riduzione Parametrica}
Se consideriamo un pattern di ordine $k$ con uno stabilizzatore di ordine $S$, il numero totale di configurazioni teoriche $3^k$ può essere ridotto drasticamente. Il numero di configurazioni \textit{effettive} da memorizzare nel PDB è approssimativamente:
\begin{equation}
    N_{eff} \approx \frac{3^k}{|\text{Stab}(\mathcal{P})|}
\end{equation}
Questa scomposizione in classi di equivalenza non solo ottimizza l'occupazione di memoria (riducendo le cache misses), ma accelera la convergenza durante la fase di apprendimento dei pesi, poiché la simmetria geometrica agisce come un regolarizzatore naturale del modello.

\subsection{Normalizzazione e Identificatori Canonici}
Per garantire l'univocità della rappresentazione dei pattern, il sistema adotta una regola di \textbf{normalizzazione canonica}. Data un'orbita $\text{Orb}(\mathcal{M})$, l'identificatore univoco del pattern $\mathcal{M}_{id}$ è definito come:
\begin{equation}
    \mathcal{M}_{id} = \min_{T \in D_4} \{ T(\mathcal{M}) \}
\end{equation}
dove l'operatore di minimo è calcolato sul valore numerico della bitmask (priorità al bit meno significativo).

Questa scelta metodologica riduce i 10 sottogruppi teorici di $D_4$ a \textbf{8 classi di equivalenza} effettive. In particolare, le distinzioni tra simmetrie puramente orizzontali e verticali (o tra le due diagonali) vengono eliminate, poiché i pattern appartenenti a tali classi vengono ricondotti allo stesso rappresentante canonico. Tale approccio garantisce che non esistano nel database due pattern che siano l'uno la rotazione o la riflessione dell'altro, ottimizzando lo spazio di ricerca e la coerenza dei dati.
\subsection{Classificazione e Normalizzazione}

Come discusso, la regola di normalizzazione canonica riduce le 10 classi teoriche a 8 classi effettive gestite dal software. Ad esempio, i pattern con simmetria orizzontale (Tipo 3) vengono mappati sulla classe verticale (Tipo 4). La Tabella \ref{tab:pattern_classification} riassume questa mappatura completa.

\begin{table}[h]
\centering
\caption{Classificazione Completa dei Pattern: dai 10 Tipi Teorici alle Classi Effettive.}
\label{tab:pattern_classification}
\begin{tabular}{llcccll}
\hline
\textbf{Tipo} & \textbf{Simmetria} & \textbf{$|Stab|$} & \textbf{$|Orb|$} & \textbf{Effettiva} & \textbf{Note} & \textbf{Pattern} \\ \hline
1 & Asimmetrico & 1 & 8 & Sì & - & \texttt{ELLE}, \texttt{2X5COR} \\
2 & Rotazionale $180^\circ$ & 2 & 4 & Sì & - & \texttt{ZSHAPE} \\
3 & Assiale Orizz. & 2 & 4 & No & $\to$ Tipo 4 & - \\
4 & Assiale Vert. & 2 & 4 & Sì & Canonica & \texttt{EDGE}, \texttt{CASTLE} \\
5 & Diagonale P & 2 & 4 & Sì & Canonica & \texttt{MACE} \\
6 & Diagonale A & 2 & 4 & No & $\to$ Tipo 5 & (\texttt{DIAG3}, \texttt{CORNER}) \\
7 & Bi-assiale & 4 & 2 & Sì & - & \texttt{RCT2X4} \\
8 & Bi-diagonale & 4 & 2 & Sì & - & \texttt{BARBEL}, \texttt{DIAG8} \\
9 & Rotazionale $90^\circ$ & 4 & 2 & Sì & - & \texttt{COREA} \\
10 & Totale ($D_4$) & 8 & 1 & Sì & - & \texttt{FOURC}, \texttt{CORE} \\
\hline
\end{tabular}
\end{table}

\subsection{Definizione Algebrica di Pattern e Istanza}

Per formalizzare il concetto di pattern nel gioco del Reversi, dobbiamo distinguere tra l'operatore geometrico e l'oggetto su cui esso agisce.

\subsection{Lo Spazio della Scacchiera}
Sia $X = \{0, 1, \dots, 63\}$ l'insieme degli indici delle celle di una scacchiera $8 \times 8$. Definiamo una \textbf{Maschera} $\mathcal{M}$ come un sottoinsieme di $X$ (o equivalentemente come un elemento dello spazio dei bit $\mathbb{B}^{64}$):
\begin{equation}
    \mathcal{M} = \{s_1, s_2, \dots, s_k\} \subseteq X
\end{equation}
dove $k$ è il numero di quadrati (celle) che compongono il pattern.

\subsection{Azione del Gruppo e Orbita}
Il Gruppo Diedrale $D_4$ agisce sullo spazio $X$ attraverso le trasformazioni $T \in G$. L'azione del gruppo sulla maschera $\mathcal{M}$ genera un insieme di nuove maschere, definito come \textbf{Orbita}:
\begin{equation}
    \text{Orb}(\mathcal{M}) = \{ T(\mathcal{M}) \mid T \in D_4 \}
\end{equation}
In questo contesto:
\begin{itemize}
    \item Il \textbf{Pattern} è l'oggetto astratto definito dall'intera orbita $\text{Orb}(\mathcal{M})$. Due maschere appartengono allo stesso pattern se e solo se esiste una trasformazione $T \in D_4$ che le rende sovrapponibili.
    \item Un' \textbf{Istanza} (o \textit{Instance}) è un singolo elemento $m \in \text{Orb}(\mathcal{M})$. Rappresenta una specifica disposizione fisica del pattern sulla scacchiera.
\end{itemize}

\subsection{Classi di Equivalenza}
L'azione di $D_4$ definisce una \textbf{relazione di equivalenza} $\sim$ sullo spazio delle maschere. Diciamo che $\mathcal{M}_a \sim \mathcal{M}_b$ se esse appartengono alla stessa orbita. 
Questa scomposizione è fondamentale per l'efficienza computazionale: invece di apprendere i parametri per ogni possibile disposizione di celle, il modello apprende le caratteristiche del \textit{Pattern} (la classe di equivalenza) e le applica a tutte le sue \textit{Istanze} (gli elementi dell'orbita).

\subsection{Lo Stabilizzatore e le Invarianze}
Per ogni maschera $\mathcal{M}$, esiste un sottogruppo di $D_4$ che non ne muta la configurazione spaziale, chiamato \textbf{Stabilizzatore}:
\begin{equation}
    \text{Stab}(\mathcal{M}) = \{ T \in D_4 \mid T(\mathcal{M}) = \mathcal{M} \}
\end{equation}
Gli elementi dello stabilizzatore (esclusa l'identità) sono gli \textbf{Automorfismi} del pattern. Essi indicano che il pattern possiede simmetrie intrinseche (ad esempio, una riflessione speculare che lo lascia invariato). Tali simmetrie permettono una ulteriore riduzione dei parametri del modello attraverso vincoli di uguaglianza sui pesi delle celle simmetriche.

\subsection{Spazio delle Configurazioni}
Data una maschera $\mathcal{M}$ di ordine $k = |\mathcal{M}|$, definiamo \textbf{Configurazione} $\mathcal{C}$ una funzione che associa ogni cella della maschera a uno stato $s \in \{0, 1, 2\}$ (rappresentanti rispettivamente cella vuota, disco nero o disco bianco). 

L'insieme di tutte le possibili configurazioni $\mathcal{S}$ ha cardinalità:
\begin{equation}
    |\mathcal{S}| = 3^k
\end{equation}

\subsection{Symmetries of Geometry vs. Symmetries of State}
È fondamentale distinguere tra l'azione del gruppo $D_4$ sulla geometria e sulla configurazione:
\begin{enumerate}
    \item \textbf{Invarianza Geometrica:} Si verifica quando $T(\mathcal{M}) = \mathcal{M}$. Lo stabilizzatore $\text{Stab}(\mathcal{M})$ definisce quali trasformazioni mantengono il pattern nella stessa posizione fisica, pur permutandone le celle interne.
    \item \textbf{Invarianza di Configurazione (Simmetria di Stato):} Una configurazione $\mathcal{C}$ si dice simmetrica sotto l'azione di $T \in \text{Stab}(\mathcal{M})$ se, dopo la permutazione delle celle operata da $T$, lo stato risultante rimane invariato. 
\end{enumerate}

Questa distinzione permette di implementare la \textbf{riduzione dei parametri}: se due diverse configurazioni $\mathcal{C}_1$ e $\mathcal{C}_2$ sono collegate da un automorfismo geometrico (cioè $\mathcal{C}_2 = T(\mathcal{C}_1)$ con $T \in \text{Stab}(\mathcal{M})$), esse devono necessariamente condividere lo stesso valore di valutazione nel database dei pattern (PDB).
\subsection{Mappatura delle Celle e Anti-trasformazioni}

Per ogni istanza del pattern nell'orbita $\text{Orb}(\mathcal{M})$, è necessario definire una corrispondenza biunivoca tra le celle dell'istanza trasformata e quelle della maschera principale $\mathcal{M}_0$.

\subsection{L'Inversa Geometrica}
Sia $T_i \in D_4$ la trasformazione che genera l'istanza $i$-esima: $\mathcal{M}_i = T_i(\mathcal{M}_0)$. Per recuperare l'informazione semantica corretta durante la fase di valutazione, definiamo l'operatore di \textbf{Anti-trasformazione} $T_i^{-1}$. 

L'azione di $T_i^{-1}$ garantisce che ogni cella $s \in \mathcal{M}_i$ venga mappata nuovamente nella sua posizione originale in $\mathcal{M}_0$:
\begin{equation}
    T_i^{-1}(\mathcal{M}_i) = \mathcal{M}_0
\end{equation}

\subsection{Recupero della Configurazione Canonica}
Sia $V$ un vettore di valori (stati delle celle) estratti dalla scacchiera in corrispondenza della maschera $\mathcal{M}_i$. L'indice di accesso al Pattern Database (PDB) deve essere calcolato sulla \textbf{configurazione normalizzata} $\mathcal{C}_{norm}$:
\begin{equation}
    \mathcal{C}_{norm} = \text{Pack}(T_i^{-1}(\text{Unpack}(V)))
\end{equation}
dove l'operazione di \textit{Unpack} distribuisce i bit sulla scacchiera e \textit{Pack} li ricompatta secondo l'ordine canonico definito dai quadrati della maschera principale. 

\subsection{Efficienza Computazionale}
Nella pratica del software, le funzioni di Anti-trasformazione sono pre-calcolate e memorizzate in una tabella di lookup. Questo permette di trasformare l'estrazione di un pattern da un'operazione geometrica complessa a una semplice permutazione di bit, mantenendo le prestazioni necessarie per l'esplorazione dell'albero di ricerca.

\section{Definizione e Calcolo dei Pattern}

Un \texttt{PATTERN} è un insieme di celle.
Ad esempio \texttt{EDGE} è il nome assegnato alle otto caselle di un bordo della board. Ci sono quattro EDGE in una board,
ottenuti ruotando di 0, 90, 180, 270 gradi il pattern di base, per convenzione quello sul bordo a nord.

Il pattern \texttt{EDGE} è quindi definito dalle caselle \texttt{\{A1, B1, C1, D1, E1, F1, G1, H1\}}, la seconda istanza del pattern è situata
sul bordo est \texttt{\{H1, H2, H3, H4, H5, H6, H7, H8\}}, la terza sul bordo sud \texttt{\{H8, G8, F8, E8, D8, C8, B8, A8\}},
la quarta su quello ovest \texttt{\{A8, A7, A6, A5, A4, A3, A2, A1\}}.
Le quattro istanze del pattern possono essere portate sulla rappresentazione di base applicando le quattro rotazioni antiorarie.
Dalla configurazione di base invece posso riportare il pattern alla sua istanza con le rotazioni orarie corrispondenti.
Il pattern può anche essere definito con la maschera dei bit sulla bitboard, uno \texttt{SquareSet} quindi che
rappresentato in esadecimale vale \texttt{0x00000000000000FF}.

Un secondo pattern che prendiamo in considerazione è \texttt{R2}, definito in esadecimale come \texttt{0x000000000000FF00}.

I pattern sono potenzialmente tantissimi, per la massima generalità focalizziamoci su un pattern che copra tutte le casistiche di queste
astrazioni ma che sia al contempo semplice, lo definiamo con il set di quattro elementi \texttt{\{A1, B1, C1, A2\}} equivalente alla maschera
\texttt{0x0000000000000107} e lo chiamiamo \texttt{ELLE}.

% WE ARE HERE

I pattern sono potenzialmente tantissimi, ma per ora soffermiamoci su questi due \texttt{EDGE} ed \texttt{R2}, questi infatti sono sufficienti
ad introdurre tutta la complessità che dobbiamo affrontare, senza perdere di generalità.

Il pattern \texttt{EDGE}, ed analogamente \texttt{R2}, è caratterizzato da otto celle.
Ogni cella ha tre configurazioni possibili: vuota, occupata da un disco nero, occupata da un disco bianco.
Diamoci la convenzione che \texttt{BLACK} significa \texttt{PLAYER}, e \texttt{WHITE} \texttt{OPPONENT}.
Le configurazioni che un \texttt{EDGE} può assumere sono quindi tre elevato alla otto quindi $3^8 = 6561$, seimilacinquecentosessantuno.
Dobbiamo definire un encoding di queste configurazioni, in maniera da assegnargli un nome univoco.
Questo nome lo chiamiamo \texttt{INDEX}.

Per calcolare il valore dell'indice $i$ nell'equazione~\ref{eq:calcolo_indice}, quindi procediamo come segue.
Ognuno dei tre colori che può assumere una cella ha un valore $c$ assegnato.
Assegniamo il valore zero alla cella vuota, il valore uno alla cella occupata dal nero, ed il valore due a quella occupata dal bianco.
Assegniamo un ordine univoco alle celle di un pattern, dalla meno significativa alla più significativa, e chiamiamo $p$ l'ordinale di questa posizione.
Quindi per l'\texttt{EDGE} la casella \texttt{A1} è la zero, \texttt{B1} è la uno, e la \texttt{H1} è la sette, che è l'ultima e quindi la $n-esima$.
L'indice di una configurazione è realizzato sommando il contributo di ciascuna cella del pattern,
con valore $[0..2]$ a secondo del colore (empty=0, black=1, white=2)
moltiplicato per tre elevato all'esponente ottenuto dalla posizione ordinale $p$ della cella nel pattern.


\begin{equation}
  i = \sum_{p=0}^{n} c \cdot 3^p
  \label{eq:calcolo_indice}
\end{equation}

Per colcolare i valori dei quattro indici del pattern \texttt{EDGE}, ma anologamente poi per il pattern \texttt{R2} od ogni altro pattern, si procede con questa sequenza
di operazioni. Viene prima ruotata la board, poi si esegue l'operazione di mask e di pack, quindi viene calcolato l'indice con la~\ref{eq:calcolo_indice}.

Per la massima generalità le istanze di un pattern nella bord possono essere fino ad otto, questo nel caso in cui il pattern non abbia una simmetria rispetto all'asse verticale
della board. In questo esempio basato sui due pattern \texttt{EDGE} ed \texttt{R2} la condizione non si presenta, non ci espone comunque questa semplificazione a perdere di generalità
in tutti i passaggi seguenti per la definizione del modello matematico della funzione di valutazione del valore del gioco.

\begin{lstlisting}[language=C]

  SquareSet
  board_pattern_pack_edge (SquareSet s)
  {
    return s & 0x00000000000000ff;
  }

  SquareSet
  board_pattern_unpack_edge (SquareSet s)
  {
    return s & 0x00000000000000ff;
  }

  SquareSet
  board_pattern_pack_r2 (SquareSet s)
  {
    return (s >> 8) & 0x00000000000000ff;
  }

  SquareSet
  board_pattern_unpack_r2 (SquareSet s)
  {
    return (s & 0x00000000000000ff) << 8;
  }
  
\end{lstlisting}


\begin{lstlisting}[language=C]

  void
  board_pattern_compute_indexes (board_pattern_index_t *indexes,
                                 const board_pattern_t *const p,
                                 const board_t *const b)
  {
    board_t tb[8];
    SquareSet m, o, c;
    board_trans_f tf;
    SquareSet mover_pattern_packed[8];
    SquareSet opponent_pattern_packed[8];

    /*                       0  1   2   3    4    5     6     7      8      9      10      11      12        13       14        15 */
    const uint64_t cim[] = { 1, 3,  9, 27,  81, 243,  729, 2187,  6561, 19683,  59049, 177147,  531441, 1594323, 4782969, 14348907 };
    const uint64_t cio[] = { 2, 6, 18, 54, 162, 486, 1458, 4374, 13122, 39366, 118098, 354294, 1062882, 3188646, 9565938, 28697814 };

    /*
    * This part should go out of the specific pattern, and done once for all patterns.
    * For now we focus on EDGE pattern alone.
    */

    m = board_get_mover_square_set(b);
    o = board_get_opponent_square_set(b);

    for (int i = 0; i < p->n_instances; i++) {
      tf = p->trans_to_principal_f[i];
      board_set_square_sets(tb + i, tf(m), tf(o));
    }

    for (int i = 0; i < p->n_instances; i++) {
      mover_pattern_packed[i] = p->pattern_pack_f(board_get_mover_square_set(tb + i));
      opponent_pattern_packed[i] = p->pattern_pack_f(board_get_opponent_square_set(tb + i));
    }

    for (int i = 0; i < p->n_instances; i++) {
      indexes[i] = 0;
      for (int j = 0; j < p->n_squares; j++) {
        c = 1ULL << j;
        indexes[i] += (c & mover_pattern_packed[i]   ) ? cim[j] : 0;
        indexes[i] += (c & opponent_pattern_packed[i]) ? cio[j] : 0;
      }
    }
    
\end{lstlisting}

\section{La funzione di valutazione della posizione (evaluation function)}

L'obiettivo della funzione di valutazione (Ef) è fornire una stima del risultato finale della partita a partire da una posizione non terminale. Nell'ecosistema Reversi, questo viene ottenuto attraverso un modello di apprendimento supervisionato che mappa le configurazioni della board su un valore atteso del game.

\subsection{Il Modello Additivo a Pattern}

Il modello adottato si basa sull'ipotesi che il valore di una posizione possa essere approssimato dalla somma dei contributi locali di diverse porzioni della scacchiera, chiamate pattern. Dato un insieme di pattern definiti $\mathcal{P}$ (come \texttt{EDGE}, \texttt{CORNER}, \texttt{DIAG8}), il valore di valutazione $V(B)$ di una board $B$ è calcolato come:

\begin{equation}
    V(B) = \sum_{P \in \mathcal{P}} \sum_{j=1}^{N_P} w_{P, i_{P,j}(B)}
\end{equation}

Dove:
\begin{itemize}
    \item $N_P$ è il numero di istanze uniche del pattern $P$ sulla scacchiera (es. 4 per \texttt{EDGE}).
    \item $i_{P,j}(B)$ è l'indice della configurazione della $j$-esima istanza del pattern $P$, calcolato in base 3 come descritto nella sezione precedente.
    \item $w_{P, k}$ è il peso (parametro del modello) associato alla $k$-esima configurazione del pattern $P$.
\end{itemize}

\subsection{Apprendimento dei Pesi e Regressione Logistica}

I pesi $w$ sono i parametri che il modello deve apprendere. Questo processo avviene utilizzando il dataset REGAB, che contiene milioni di posizioni risolte esattamente. 

Il modello può essere interpretato in due modi:
\begin{enumerate}
    \item \textbf{Regressione Lineare:} Il valore $V(B)$ predice direttamente il differenziale di dischi finale $[-64, +64]$.
    \item \textbf{Regressione Logistica:} La somma dei pesi rappresenta il \textit{logit} di un modello logistico che predice la probabilità di vittoria $P(vittoria) = \sigma(V(B))$, dove $\sigma$ è la funzione sigmoide.
\end{enumerate}

L'approccio logistico è particolarmente efficace nel Reversi, poiché la natura del gioco è fortemente non lineare: piccoli vantaggi posizionali possono tradursi in vittorie schiaccianti o sconfitte improvvise (specialmente nelle fasi finali).

\subsection{Riduzione Parametrica e Generalizzazione}

Grazie alla teoria dei gruppi discussa nel Capitolo \ref{sec:gruppo_diedrale}, il numero di parametri da apprendere è drasticamente ridotto. Se un pattern ha uno stabilizzatore di ordine $S$, il numero di configurazioni uniche è $3^k/S$. 

Inoltre, applicando la \textbf{normalizzazione canonica}, tutte le istanze simmetriche di un pattern condividono la stessa tabella di pesi (Pattern Database - PDB). Questo non solo riduce l'occupazione di memoria, ma garantisce una migliore generalizzazione del modello: l'esperienza appresa su un angolo della scacchiera viene automaticamente applicata a tutti e quattro gli angoli grazie alla simmetria del gruppo $D_4$.

\subsection{Verso il Modello Neural Network (RNNM)}

L'attuale architettura a "Two-Layer Model" prevede un primo livello di estrazione di feature (gli indici dei pattern) e un secondo livello di combinazione lineare (la Ef). L'evoluzione naturale di questo sistema è il \textbf{Reversi Neural Network Model (RNNM)}, dove i contributi dei pattern non vengono semplicemente sommati, ma processati attraverso strati nascosti (hidden layers) per catturare interazioni di ordine superiore tra diverse zone della scacchiera.

\newpage
\appendix
\section{Catalogo dei Pattern}

In questa appendice vengono visualizzati i pattern principali utilizzati dal sistema. Per ogni pattern sono mostrate le 8 trasformazioni del gruppo $D_4$. I numeri all'interno delle celle indicano l'indice di packing progressivo (0, 1, 2, ...) all'interno del pattern, permettendo di tracciare visivamente la permutazione delle celle sotto ogni trasformazione.

\subsection{Pattern ELLE}

\begin{lstlisting}[language=bash, basicstyle=\ttfamily\tiny]
[Pattern: name = ELLE, mask = 0x0000000000000107]
  [n_squares = 4, n_configurations = 81, n_instances = 8, n_stabilizer = 1]
  Cells:               [A1, B1, C1, A2]
  Transformed masks:   
    0x0000000000000107, 0x00000000008080C0, 0xE080000000000000, 0x0301010000000000, 
    0x00000000000080E0, 0xC080800000000000, 0x0701000000000000, 0x0000000000010103
  Mask indexes:        [0, 1, 2, 3, 4, 5, 6, 7]
  Unique masks:        
    0x0000000000000107, 0x00000000008080C0, 0xE080000000000000, 0x0301010000000000, 
    0x00000000000080E0, 0xC080800000000000, 0x0701000000000000, 0x0000000000010103
  Unique mask indexes: [0, 1, 2, 3, 4, 5, 6, 7]
  Transf. functions:   [Id, Rot90, Vert, DiagP]
  Anti-transf. f.:     [Id, Rot270, Vert, DiagP]
  Symmetry functions:  []
\end{lstlisting}

\begin{figure}[htbp]
    \centering
    \begin{subfigure}[b]{0.23\textwidth}
        \centering
        \scalebox{0.45}{
            \begin{othelloboard}{1}
                \posannotation{a1}{0} \posannotation{b1}{1} \posannotation{c1}{2} \posannotation{a2}{3}
            \end{othelloboard}
        }
        \caption{$T_0$ : Id \\ \fontsize{3}{4}\selectfont \texttt{0000000000000107}}
    \end{subfigure}
    \hfill
    \begin{subfigure}[b]{0.23\textwidth}
        \centering
        \scalebox{0.45}{
            \begin{othelloboard}{1}
                \posannotation{h1}{0} \posannotation{h2}{1} \posannotation{h3}{2} \posannotation{g1}{3}
            \end{othelloboard}
        }
        \caption{$T_1$ : Rot90 \\ \fontsize{3}{4}\selectfont \texttt{00000000008080C0}}
    \end{subfigure}
    \hfill
    \begin{subfigure}[b]{0.23\textwidth}
        \centering
        \scalebox{0.45}{
            \begin{othelloboard}{1}
                \posannotation{h8}{0} \posannotation{g8}{1} \posannotation{f8}{2} \posannotation{h7}{3}
            \end{othelloboard}
        }
        \caption{$T_2$ : Rot180 \\ \fontsize{3}{4}\selectfont \texttt{E080000000000000}}
    \end{subfigure}
    \hfill
    \begin{subfigure}[b]{0.23\textwidth}
        \centering
        \scalebox{0.45}{
            \begin{othelloboard}{1}
                \posannotation{a8}{0} \posannotation{a7}{1} \posannotation{a6}{2} \posannotation{b8}{3}
            \end{othelloboard}
        }
        \caption{$T_3$ : Rot270 \\ \fontsize{3}{4}\selectfont \texttt{0301010000000000}}
    \end{subfigure}

    \vspace{0.3cm}

    \begin{subfigure}[b]{0.23\textwidth}
        \centering
        \scalebox{0.45}{
            \begin{othelloboard}{1}
                \posannotation{h1}{0} \posannotation{g1}{1} \posannotation{f1}{2} \posannotation{h2}{3}
            \end{othelloboard}
        }
        \caption{$T_4$ : Vert \\ \fontsize{3}{4}\selectfont \texttt{00000000000080E0}}
    \end{subfigure}
    \hfill
    \begin{subfigure}[b]{0.23\textwidth}
        \centering
        \scalebox{0.45}{
            \begin{othelloboard}{1}
                \posannotation{h8}{0} \posannotation{h7}{1} \posannotation{h6}{2} \posannotation{g8}{3}
            \end{othelloboard}
        }
        \caption{$T_5$ : DiagP \\ \fontsize{3}{4}\selectfont \texttt{C080800000000000}}
    \end{subfigure}
    \hfill
    \begin{subfigure}[b]{0.23\textwidth}
        \centering
        \scalebox{0.45}{
            \begin{othelloboard}{1}
                \posannotation{a8}{0} \posannotation{b8}{1} \posannotation{c8}{2} \posannotation{a7}{3}
            \end{othelloboard}
        }
        \caption{$T_6$ : Horiz \\ \fontsize{3}{4}\selectfont \texttt{0701000000000000}}
    \end{subfigure}
    \hfill
    \begin{subfigure}[b]{0.23\textwidth}
        \centering
        \scalebox{0.45}{
            \begin{othelloboard}{1}
                \posannotation{a1}{0} \posannotation{a2}{1} \posannotation{a3}{2} \posannotation{b1}{3}
            \end{othelloboard}
        }
        \caption{$T_7$ : DiagA \\ \fontsize{3}{4}\selectfont \texttt{0000000000010103}}
    \end{subfigure}
    \caption{Istanze e indici di permutazione del pattern ELLE.}
\end{figure}

\newpage
\subsection{Pattern EDGE}

\begin{lstlisting}[language=bash, basicstyle=\ttfamily\tiny]
[Pattern: name = EDGE, mask = 0x00000000000000ff]
  [n_squares = 8, n_configurations = 6561, n_instances = 4, n_stabilizer = 2]
  Cells:               [A1, B1, C1, D1, E1, F1, G1, H1]
  Transformed masks:   
    0x00000000000000FF, 0x8080808080808080, 0xFF00000000000000, 0x0101010101010101, 
    0x00000000000000FF, 0x8080808080808080, 0xFF00000000000000, 0x0101010101010101
  Mask indexes:        [0, 1, 2, 3, 0, 1, 2, 3]
  Unique masks:        
    0x00000000000000FF, 0x8080808080808080, 0xFF00000000000000, 0x0101010101010101
  Unique mask indexes: [0, 1, 2, 3]
  Transf. functions:   [Id, Rot90, Rot180, Rot270]
  Anti-transf. f.:     [Id, Rot270, Rot180, Rot90]
  Symmetry functions:  [Vert]
\end{lstlisting}

\begin{figure}[htbp]
    \centering
    \begin{subfigure}[b]{0.23\textwidth}
        \centering
        \scalebox{0.45}{
            \begin{othelloboard}{1}
                \posannotation{a1}{0} \posannotation{b1}{1} \posannotation{c1}{2} \posannotation{d1}{3} \posannotation{e1}{4} \posannotation{f1}{5} \posannotation{g1}{6} \posannotation{h1}{7}
            \end{othelloboard}
        }
        \caption{$T_0$ : Id \\ \fontsize{3}{4}\selectfont \texttt{00000000000000FF}}
    \end{subfigure}
    \hfill
    \begin{subfigure}[b]{0.23\textwidth}
        \centering
        \scalebox{0.45}{
            \begin{othelloboard}{1}
                \posannotation{h1}{0} \posannotation{h2}{1} \posannotation{h3}{2} \posannotation{h4}{3} \posannotation{h5}{4} \posannotation{h6}{5} \posannotation{h7}{6} \posannotation{h8}{7}
            \end{othelloboard}
        }
        \caption{$T_1$ : Rot90 \\ \fontsize{3}{4}\selectfont \texttt{8080808080808080}}
    \end{subfigure}
    \hfill
    \begin{subfigure}[b]{0.23\textwidth}
        \centering
        \scalebox{0.45}{
            \begin{othelloboard}{1}
                \posannotation{h8}{0} \posannotation{g8}{1} \posannotation{f8}{2} \posannotation{e8}{3} \posannotation{d8}{4} \posannotation{c8}{5} \posannotation{b8}{6} \posannotation{a8}{7}
            \end{othelloboard}
        }
        \caption{$T_2$ : Rot180 \\ \fontsize{3}{4}\selectfont \texttt{FF00000000000000}}
    \end{subfigure}
    \hfill
    \begin{subfigure}[b]{0.23\textwidth}
        \centering
        \scalebox{0.45}{
            \begin{othelloboard}{1}
                \posannotation{a8}{0} \posannotation{a7}{1} \posannotation{a6}{2} \posannotation{a5}{3} \posannotation{a4}{4} \posannotation{a3}{5} \posannotation{a2}{6} \posannotation{a1}{7}
            \end{othelloboard}
        }
        \caption{$T_3$ : Rot270 \\ \fontsize{3}{4}\selectfont \texttt{0101010101010101}}
    \end{subfigure}

    \vspace{0.3cm}

    \begin{subfigure}[b]{0.23\textwidth}
        \centering
        \scalebox{0.45}{
            \begin{othelloboard}{1}
                \posannotation{h1}{0} \posannotation{g1}{1} \posannotation{f1}{2} \posannotation{e1}{3} \posannotation{d1}{4} \posannotation{c1}{5} \posannotation{b1}{6} \posannotation{a1}{7}
            \end{othelloboard}
        }
        \caption{$T_4$ : Vert \\ \fontsize{3}{4}\selectfont \texttt{00000000000000FF}}
    \end{subfigure}
    \hfill
    \begin{subfigure}[b]{0.23\textwidth}
        \centering
        \scalebox{0.45}{
            \begin{othelloboard}{1}
                \posannotation{h8}{0} \posannotation{h7}{1} \posannotation{h6}{2} \posannotation{h5}{3} \posannotation{h4}{4} \posannotation{h3}{5} \posannotation{h2}{6} \posannotation{h1}{7}
            \end{othelloboard}
        }
        \caption{$T_5$ : DiagP \\ \fontsize{3}{4}\selectfont \texttt{8080808080808080}}
    \end{subfigure}
    \hfill
    \begin{subfigure}[b]{0.23\textwidth}
        \centering
        \scalebox{0.45}{
            \begin{othelloboard}{1}
                \posannotation{a8}{0} \posannotation{b8}{1} \posannotation{c8}{2} \posannotation{d8}{3} \posannotation{e8}{4} \posannotation{f8}{5} \posannotation{g8}{6} \posannotation{h8}{7}
            \end{othelloboard}
        }
        \caption{$T_6$ : Horiz \\ \fontsize{3}{4}\selectfont \texttt{FF00000000000000}}
    \end{subfigure}
    \hfill
    \begin{subfigure}[b]{0.23\textwidth}
        \centering
        \scalebox{0.45}{
            \begin{othelloboard}{1}
                \posannotation{a1}{0} \posannotation{a2}{1} \posannotation{a3}{2} \posannotation{a4}{3} \posannotation{a5}{4} \posannotation{a6}{5} \posannotation{a7}{6} \posannotation{a8}{7}
            \end{othelloboard}
        }
        \caption{$T_7$ : DiagA \\ \fontsize{3}{4}\selectfont \texttt{0101010101010101}}
    \end{subfigure}
    \caption{Istanze e indici di permutazione del pattern EDGE.}
\end{figure}

\newpage
\subsection{Pattern ZSHAPE}

\begin{lstlisting}[language=bash, basicstyle=\ttfamily\tiny]
[Pattern: name = ZSHAPE, mask = 0x0000000c30000000]
  [n_squares = 4, n_configurations = 81, n_instances = 4, n_stabilizer = 2]
  Cells:               [E4, F4, C5, D5]
  Transformed masks:   
    0x0000000C30000000, 0x0000101008080000, 0x0000000C30000000, 0x0000101008080000, 
    0x000000300C000000, 0x0000080810100000, 0x000000300C000000, 0x0000080810100000
  Mask indexes:        [0, 1, 0, 1, 4, 5, 4, 5]
  Unique masks:        
    0x0000000C30000000, 0x0000101008080000, 0x000000300C000000, 0x0000080810100000
  Unique mask indexes: [0, 1, 4, 5]
  Transf. functions:   [Id, Rot90, Vert, DiagP]
  Anti-transf. f.:     [Id, Rot270, Vert, DiagP]
  Symmetry functions:  [Rot180]
\end{lstlisting}

\begin{figure}[htbp]
    \centering
    \begin{subfigure}[b]{0.23\textwidth}
        \centering
        \scalebox{0.45}{
            \begin{othelloboard}{1}
                \posannotation{e4}{0} \posannotation{f4}{1} \posannotation{c5}{2} \posannotation{d5}{3}
            \end{othelloboard}
        }
        \caption{$T_0$ : Id \\ \fontsize{3}{4}\selectfont \texttt{0000000C30000000}}
    \end{subfigure}
    \hfill
    \begin{subfigure}[b]{0.23\textwidth}
        \centering
        \scalebox{0.45}{
            \begin{othelloboard}{1}
                \posannotation{e5}{0} \posannotation{e6}{1} \posannotation{d3}{2} \posannotation{d4}{3}
            \end{othelloboard}
        }
        \caption{$T_1$ : Rot90 \\ \fontsize{3}{4}\selectfont \texttt{0000101008080000}}
    \end{subfigure}
    \hfill
    \begin{subfigure}[b]{0.23\textwidth}
        \centering
        \scalebox{0.45}{
            \begin{othelloboard}{1}
                \posannotation{d5}{0} \posannotation{c5}{1} \posannotation{f4}{2} \posannotation{e4}{3}
            \end{othelloboard}
        }
        \caption{$T_2$ : Rot180 \\ \fontsize{3}{4}\selectfont \texttt{0000000C30000000}}
    \end{subfigure}
    \hfill
    \begin{subfigure}[b]{0.23\textwidth}
        \centering
        \scalebox{0.45}{
            \begin{othelloboard}{1}
                \posannotation{d4}{0} \posannotation{d3}{1} \posannotation{e6}{2} \posannotation{e5}{3}
            \end{othelloboard}
        }
        \caption{$T_3$ : Rot270 \\ \fontsize{3}{4}\selectfont \texttt{0000101008080000}}
    \end{subfigure}

    \vspace{0.3cm}

    \begin{subfigure}[b]{0.23\textwidth}
        \centering
        \scalebox{0.45}{
            \begin{othelloboard}{1}
                \posannotation{d4}{0} \posannotation{c4}{1} \posannotation{f5}{2} \posannotation{e5}{3}
            \end{othelloboard}
        }
        \caption{$T_4$ : Vert \\ \fontsize{3}{4}\selectfont \texttt{000000300C000000}}
    \end{subfigure}
    \hfill
    \begin{subfigure}[b]{0.23\textwidth}
        \centering
        \scalebox{0.45}{
            \begin{othelloboard}{1}
                \posannotation{e4}{0} \posannotation{e3}{1} \posannotation{d6}{2} \posannotation{d5}{3}
            \end{othelloboard}
        }
        \caption{$T_5$ : DiagP \\ \fontsize{3}{4}\selectfont \texttt{0000080810100000}}
    \end{subfigure}
    \hfill
    \begin{subfigure}[b]{0.23\textwidth}
        \centering
        \scalebox{0.45}{
            \begin{othelloboard}{1}
                \posannotation{e5}{0} \posannotation{f5}{1} \posannotation{c4}{2} \posannotation{d4}{3}
            \end{othelloboard}
        }
        \caption{$T_6$ : Horiz \\ \fontsize{3}{4}\selectfont \texttt{000000300C000000}}
    \end{subfigure}
    \hfill
    \begin{subfigure}[b]{0.23\textwidth}
        \centering
        \scalebox{0.45}{
            \begin{othelloboard}{1}
                \posannotation{d5}{0} \posannotation{d6}{1} \posannotation{e3}{2} \posannotation{e4}{3}
            \end{othelloboard}
        }
        \caption{$T_7$ : DiagA \\ \fontsize{3}{4}\selectfont \texttt{0000080810100000}}
    \end{subfigure}
    \caption{Istanze e indici di permutazione del pattern ZSHAPE.}
\end{figure}

\newpage
\subsection{Pattern RCT2X4}

\begin{lstlisting}[language=bash, basicstyle=\ttfamily\tiny]
[Pattern: name = RCT2X4, mask = 0x0000003c3c000000]
  [n_squares = 8, n_configurations = 6561, n_instances = 2, n_stabilizer = 4]
  Cells:               [C4, D4, E4, F4, C5, D5, E5, F5]
  Transformed masks:   
    0x0000003C3C000000, 0x0000181818180000, 0x0000003C3C000000, 0x0000181818180000, 
    0x0000003C3C000000, 0x0000181818180000, 0x0000003C3C000000, 0x0000181818180000
  Mask indexes:        [0, 1, 0, 1, 0, 1, 0, 1]
  Unique masks:        
    0x0000003C3C000000, 0x0000181818180000
  Unique mask indexes: [0, 1]
  Transf. functions:   [Id, Rot90]
  Anti-transf. f.:     [Id, Rot270]
  Symmetry functions:  [Rot180, Vert, Horiz]
\end{lstlisting}

\begin{figure}[htbp]
    \centering
    \begin{subfigure}[b]{0.23\textwidth}
        \centering
        \scalebox{0.45}{
            \begin{othelloboard}{1}
                \posannotation{c4}{0} \posannotation{d4}{1} \posannotation{e4}{2} \posannotation{f4}{3} 
                \posannotation{c5}{4} \posannotation{d5}{5} \posannotation{e5}{6} \posannotation{f5}{7}
            \end{othelloboard}
        }
        \caption{$T_0$ : Id \\ \fontsize{3}{4}\selectfont \texttt{0000003C3C000000}}
    \end{subfigure}
    \hfill
    \begin{subfigure}[b]{0.23\textwidth}
        \centering
        \scalebox{0.45}{
            \begin{othelloboard}{1}
                \posannotation{e3}{0} \posannotation{e4}{1} \posannotation{e5}{2} \posannotation{e6}{3} 
                \posannotation{d3}{4} \posannotation{d4}{5} \posannotation{d5}{6} \posannotation{d6}{7}
            \end{othelloboard}
        }
        \caption{$T_1$ : Rot90 \\ \fontsize{3}{4}\selectfont \texttt{0000181818180000}}
    \end{subfigure}
    \hfill
    \begin{subfigure}[b]{0.23\textwidth}
        \centering
        \scalebox{0.45}{
            \begin{othelloboard}{1}
                \posannotation{f5}{0} \posannotation{e5}{1} \posannotation{d5}{2} \posannotation{c5}{3} 
                \posannotation{f4}{4} \posannotation{e4}{5} \posannotation{d4}{6} \posannotation{c4}{7}
            \end{othelloboard}
        }
        \caption{$T_2$ : Rot180 \\ \fontsize{3}{4}\selectfont \texttt{0000003C3C000000}}
    \end{subfigure}
    \hfill
    \begin{subfigure}[b]{0.23\textwidth}
        \centering
        \scalebox{0.45}{
            \begin{othelloboard}{1}
                \posannotation{d6}{0} \posannotation{d5}{1} \posannotation{d4}{2} \posannotation{d3}{3} 
                \posannotation{e6}{4} \posannotation{e5}{5} \posannotation{e4}{6} \posannotation{e3}{7}
            \end{othelloboard}
        }
        \caption{$T_3$ : Rot270 \\ \fontsize{3}{4}\selectfont \texttt{0000181818180000}}
    \end{subfigure}

    \vspace{0.3cm}

    \begin{subfigure}[b]{0.23\textwidth}
        \centering
        \scalebox{0.45}{
            \begin{othelloboard}{1}
                \posannotation{f4}{0} \posannotation{e4}{1} \posannotation{d4}{2} \posannotation{c4}{3} 
                \posannotation{f5}{4} \posannotation{e5}{5} \posannotation{d5}{6} \posannotation{c5}{7}
            \end{othelloboard}
        }
        \caption{$T_4$ : Vert \\ \fontsize{3}{4}\selectfont \texttt{0000003C3C000000}}
    \end{subfigure}
    \hfill
    \begin{subfigure}[b]{0.23\textwidth}
        \centering
        \scalebox{0.45}{
            \begin{othelloboard}{1}
                \posannotation{e6}{0} \posannotation{e5}{1} \posannotation{e4}{2} \posannotation{e3}{3} 
                \posannotation{d6}{4} \posannotation{d5}{5} \posannotation{d4}{6} \posannotation{d3}{7}
            \end{othelloboard}
        }
        \caption{$T_5$ : DiagP \\ \fontsize{3}{4}\selectfont \texttt{0000181818180000}}
    \end{subfigure}
    \hfill
    \begin{subfigure}[b]{0.23\textwidth}
        \centering
        \scalebox{0.45}{
            \begin{othelloboard}{1}
                \posannotation{c5}{0} \posannotation{d5}{1} \posannotation{e5}{2} \posannotation{f5}{3} 
                \posannotation{c4}{4} \posannotation{d4}{5} \posannotation{e4}{6} \posannotation{f4}{7}
            \end{othelloboard}
        }
        \caption{$T_6$ : Horiz \\ \fontsize{3}{4}\selectfont \texttt{0000003C3C000000}}
    \end{subfigure}
    \hfill
    \begin{subfigure}[b]{0.23\textwidth}
        \centering
        \scalebox{0.45}{
            \begin{othelloboard}{1}
                \posannotation{d3}{0} \posannotation{d4}{1} \posannotation{d5}{2} \posannotation{d6}{3} 
                \posannotation{e3}{4} \posannotation{e4}{5} \posannotation{e5}{6} \posannotation{e6}{7}
            \end{othelloboard}
        }
        \caption{$T_7$ : DiagA \\ \fontsize{3}{4}\selectfont \texttt{0000181818180000}}
    \end{subfigure}
    \caption{Istanze e indici di permutazione del pattern RCT2X4.}
\end{figure}

\newpage
\subsection{Pattern BARBEL}

\begin{lstlisting}[language=bash, basicstyle=\ttfamily\tiny]
[Pattern: name = BARBEL, mask = 0x030304081020C0C0]
  [n_squares = 12, n_configurations = 531441, n_instances = 2, n_stabilizer = 4]
  Cells:               [G1, H1, G2, H2, F3, E4, D5, C6, A7, B7, A8, B8]
  Transformed masks:   
    0x030304081020C0C0, 0xC0C0201008040303, 0x030304081020C0C0, 0xC0C0201008040303,
    0xC0C0201008040303, 0x030304081020C0C0, 0xC0C0201008040303, 0x030304081020C0C0
  Mask indexes:        [0, 1, 0, 1, 1, 0, 1, 0]
  Unique masks:        
    0x030304081020C0C0, 0xC0C0201008040303
  Unique mask indexes: [0, 1]
  Transf. functions:   [Id, Rot90]
  Anti-transf. f.:     [Id, Rot270]
  Symmetry functions:  [Rot180, DiagP, DiagA]
\end{lstlisting}

\begin{figure}[htbp]
    \centering
    \begin{subfigure}[b]{0.23\textwidth}
        \centering
        \scalebox{0.45}{
            \begin{othelloboard}{1}
                \posannotation{g1}{0} \posannotation{h1}{1} \posannotation{g2}{2} \posannotation{h2}{3}
                \posannotation{f3}{4} \posannotation{e4}{5} \posannotation{d5}{6} \posannotation{c6}{7}
                \posannotation{a7}{8} \posannotation{b7}{9} \posannotation{a8}{10} \posannotation{b8}{11}
            \end{othelloboard}
        }
        \caption{$T_0$ : Id \\ \fontsize{3}{4}\selectfont \texttt{030304081020C0C0}}
    \end{subfigure}
    \hfill
    \begin{subfigure}[b]{0.23\textwidth}
        \centering
        \scalebox{0.45}{
            \begin{othelloboard}{1}
                \posannotation{a1}{10} \posannotation{b1}{8} \posannotation{a2}{11} \posannotation{b2}{9}
                \posannotation{c3}{7} \posannotation{d4}{6} \posannotation{e5}{5} \posannotation{f6}{4}
                \posannotation{g7}{2} \posannotation{h7}{0} \posannotation{g8}{3} \posannotation{h8}{1}
            \end{othelloboard}
        }
        \caption{$T_1$ : Rot90 \\ \fontsize{3}{4}\selectfont \texttt{C0C0201008040303}}
    \end{subfigure}
    \hfill
    \begin{subfigure}[b]{0.23\textwidth}
        \centering
        \scalebox{0.45}{
            \begin{othelloboard}{1}
                \posannotation{g1}{11} \posannotation{h1}{10} \posannotation{g2}{9} \posannotation{h2}{8}
                \posannotation{f3}{7} \posannotation{e4}{6} \posannotation{d5}{5} \posannotation{c6}{4}
                \posannotation{a7}{3} \posannotation{b7}{2} \posannotation{a8}{1} \posannotation{b8}{0}
            \end{othelloboard}
        }
        \caption{$T_2$ : Rot180 \\ \fontsize{3}{4}\selectfont \texttt{030304081020C0C0}}
    \end{subfigure}
    \hfill
    \begin{subfigure}[b]{0.23\textwidth}
        \centering
        \scalebox{0.45}{
            \begin{othelloboard}{1}
                \posannotation{a1}{1} \posannotation{b1}{3} \posannotation{a2}{0} \posannotation{b2}{2}
                \posannotation{c3}{4} \posannotation{d4}{5} \posannotation{e5}{6} \posannotation{f6}{7}
                \posannotation{g7}{9} \posannotation{h7}{11} \posannotation{g8}{8} \posannotation{h8}{10}
            \end{othelloboard}
        }
        \caption{$T_3$ : Rot270 \\ \fontsize{3}{4}\selectfont \texttt{C0C0201008040303}}
    \end{subfigure}

    \vspace{0.3cm}
    
    \begin{subfigure}[b]{0.23\textwidth}
        \centering
        \scalebox{0.45}{
            \begin{othelloboard}{1}
                \posannotation{a1}{1} \posannotation{b1}{0} \posannotation{a2}{3} \posannotation{b2}{2}
                \posannotation{c3}{4} \posannotation{d4}{5} \posannotation{e5}{6} \posannotation{f6}{7}
                \posannotation{g7}{9} \posannotation{h7}{8} \posannotation{g8}{11} \posannotation{h8}{10}
            \end{othelloboard}
        }
        \caption{$T_4$ : Vert \\ \fontsize{3}{4}\selectfont \texttt{C0C0201008040303}}
    \end{subfigure}
    \hfill
    \begin{subfigure}[b]{0.23\textwidth}
        \centering
        \scalebox{0.45}{
            \begin{othelloboard}{1}
                \posannotation{g1}{3} \posannotation{h1}{1} \posannotation{g2}{2} \posannotation{h2}{0}
                \posannotation{f3}{4} \posannotation{e4}{5} \posannotation{d5}{6} \posannotation{c6}{7}
                \posannotation{a7}{11} \posannotation{b7}{9} \posannotation{a8}{10} \posannotation{b8}{8}
            \end{othelloboard}
        }
        \caption{$T_5$ : DiagP \\ \fontsize{3}{4}\selectfont \texttt{030304081020C0C0}}
    \end{subfigure}
    \hfill
    \begin{subfigure}[b]{0.23\textwidth}
        \centering
        \scalebox{0.45}{
            \begin{othelloboard}{1}
                \posannotation{a1}{10} \posannotation{b1}{11} \posannotation{a2}{8} \posannotation{b2}{9}
                \posannotation{c3}{7} \posannotation{d4}{6} \posannotation{e5}{5} \posannotation{f6}{4}
                \posannotation{g7}{2} \posannotation{h7}{3} \posannotation{g8}{0} \posannotation{h8}{1}
            \end{othelloboard}
        }
        \caption{$T_6$ : Horiz \\ \fontsize{3}{4}\selectfont \texttt{C0C0201008040303}}
    \end{subfigure}
    \hfill
    \begin{subfigure}[b]{0.23\textwidth}
        \centering
        \scalebox{0.45}{
            \begin{othelloboard}{1}
                \posannotation{g1}{8} \posannotation{h1}{10} \posannotation{g2}{9} \posannotation{h2}{11}
                \posannotation{f3}{7} \posannotation{e4}{6} \posannotation{d5}{5} \posannotation{c6}{4}
                \posannotation{a7}{0} \posannotation{b7}{2} \posannotation{a8}{1} \posannotation{b8}{3}
            \end{othelloboard}
        }
        \caption{$T_7$ : DiagA \\ \fontsize{3}{4}\selectfont \texttt{030304081020C0C0}}
    \end{subfigure}
    
    \caption{Istanze e indici di permutazione del pattern BARBEL.}
  \end{figure}
  
\newpage
\subsection{Pattern DIAG3}

\begin{lstlisting}[language=bash, basicstyle=\ttfamily\tiny]
[Pattern: name = DIAG3, mask = 0x0000000000010204]
  [n_squares = 3, n_configurations = 27, n_instances = 4, n_stabilizer = 2]
  Cells:               [C1, B2, A3]
  Transformed masks:   
    0x0000000000010204, 0x0000000000804020, 0x2040800000000000, 0x0402010000000000, 
    0x0000000000804020, 0x2040800000000000, 0x0402010000000000, 0x0000000000010204
  Mask indexes:        [0, 1, 2, 3, 1, 2, 3, 0]
  Unique masks:        
    0x0000000000010204, 0x0000000000804020, 0x2040800000000000, 0x0402010000000000
  Unique mask indexes: [0, 1, 2, 3]
  Transf. functions:   [Id, Rot90, Rot180, Rot270]
  Anti-transf. f.:     [Id, Rot270, Rot180, Rot90]
  Symmetry functions:  [DiagP]
\end{lstlisting}

\begin{figure}[htbp]
    \centering
    \begin{subfigure}[b]{0.23\textwidth}
        \centering
        \scalebox{0.45}{
            \begin{othelloboard}{1}
                \posannotation{c1}{0} \posannotation{b2}{1} \posannotation{a3}{2}
            \end{othelloboard}
        }
        \caption{$T_0$ : Id \\ \fontsize{3}{4}\selectfont \texttt{0000000000010204}}
    \end{subfigure}
    \hfill
    \begin{subfigure}[b]{0.23\textwidth}
        \centering
        \scalebox{0.45}{
            \begin{othelloboard}{1}
                \posannotation{f1}{2} \posannotation{g2}{1} \posannotation{h3}{0}
            \end{othelloboard}
        }
        \caption{$T_1$ : Rot90 \\ \fontsize{3}{4}\selectfont \texttt{0000000000804020}}
    \end{subfigure}
    \hfill
    \begin{subfigure}[b]{0.23\textwidth}
        \centering
        \scalebox{0.45}{
            \begin{othelloboard}{1}
                \posannotation{h6}{2} \posannotation{g7}{1} \posannotation{f8}{0}
            \end{othelloboard}
        }
        \caption{$T_2$ : Rot180 \\ \fontsize{3}{4}\selectfont \texttt{2040800000000000}}
    \end{subfigure}
    \hfill
    \begin{subfigure}[b]{0.23\textwidth}
        \centering
        \scalebox{0.45}{
            \begin{othelloboard}{1}
                \posannotation{a6}{0} \posannotation{b7}{1} \posannotation{c8}{2}
            \end{othelloboard}
        }
        \caption{$T_3$ : Rot270 \\ \fontsize{3}{4}\selectfont \texttt{0402010000000000}}
    \end{subfigure}

    \vspace{0.3cm}

    \begin{subfigure}[b]{0.23\textwidth}
        \centering
        \scalebox{0.45}{
            \begin{othelloboard}{1}
                \posannotation{f1}{0} \posannotation{g2}{1} \posannotation{h3}{2}
            \end{othelloboard}
        }
        \caption{$T_4$ : Vert \\ \fontsize{3}{4}\selectfont \texttt{0000000000804020}}
    \end{subfigure}
    \hfill
    \begin{subfigure}[b]{0.23\textwidth}
        \centering
        \scalebox{0.45}{
            \begin{othelloboard}{1}
                \posannotation{h6}{0} \posannotation{g7}{1} \posannotation{f8}{2}
            \end{othelloboard}
        }
        \caption{$T_5$ : DiagP \\ \fontsize{3}{4}\selectfont \texttt{2040800000000000}}
    \end{subfigure}
    \hfill
    \begin{subfigure}[b]{0.23\textwidth}
        \centering
        \scalebox{0.45}{
            \begin{othelloboard}{1}
                \posannotation{a6}{2} \posannotation{b7}{1} \posannotation{c8}{0}
            \end{othelloboard}
        }
        \caption{$T_6$ : Horiz \\ \fontsize{3}{4}\selectfont \texttt{0402010000000000}}
    \end{subfigure}
    \hfill
    \begin{subfigure}[b]{0.23\textwidth}
        \centering
        \scalebox{0.45}{
            \begin{othelloboard}{1}
                \posannotation{c1}{2} \posannotation{b2}{1} \posannotation{a3}{0}
            \end{othelloboard}
        }
        \caption{$T_6$ : DiagA \\ \fontsize{3}{4}\selectfont \texttt{0402010000000000}}
    \end{subfigure}
    
    \caption{Istanze e indici di permutazione del pattern DIAG3.}
\end{figure}

\end{document}
