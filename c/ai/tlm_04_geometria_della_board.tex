%
%  tlm_geometria_della_board.tex
%
%  Copyright (c) 2026 Roberto Corradini. All rights reserved.
%
%  This file is part of the reversi program
%  http://github.com/rcrr/reversi
%
%  This program is free software; you can redistribute it and/or modify it
%  under the terms of the GNU General Public License as published by the
%  Free Software Foundation; either version 3, or (at your option) any
%  later version.
%
%  This program is distributed in the hope that it will be useful,
%  but WITHOUT ANY WARRANTY; without even the implied warranty of
%  MERCHANTABILITY or FITNESS FOR A PARTICULAR PURPOSE. See the
%  GNU General Public License for more details.
%
%  You should have received a copy of the GNU General Public License
%  along with this program; if not, write to the Free Software
%  Foundation, Inc., 59 Temple Place - Suite 330, Boston, MA  02111-1307, USA
%  or visit the site <http://www.gnu.org/licenses/>.
%


\section{Geometria della Scacchiera: Il Gruppo Diedrale \texorpdfstring{$\bm{D_4}$}{D4}}

\subsection{Rotazioni della bitboard}

Sia gli square set che le board sono passibili di otto differenti trasformazioni, che ruotano rigidamente, oppure specchiano la bitboard rispetto gli assi di simmetria
naturali. Le trasformazioni sono poi usate per calcolare gli indici delle istanze dei diversi pattern.
La board può essere ruotata di zero, 90, 180, 270 gradi in senso orario, e può essere specchiata secondo gli assi verticale, orizontale, diagonale NW-SE e diagonale NE-SW.
Sono poi definite le operazioni delle otto antitrasformazioni che riportano la bitboard nella condizione originale.

A titolo di esempio la board:

{\tiny
\begin{verbatim}
A1 B1 C1 D1 E1 F1 G1 H1
A2 B2 C2 D2 E2 F2 G2 H2
A3 B3 C3 D3 E3 F3 G3 H3
A4 B4 C4 D4 E4 F4 G4 H4
A5 B5 C5 D5 E5 F5 G5 H5
A6 B6 C6 D6 E6 F6 G6 H6
A7 B7 C7 D7 E7 F7 G7 H7
A8 B8 C8 D8 E8 F8 G8 H8
\end{verbatim}
}

ruotata in senso orario di 90 gradi diventa:

{\tiny
\begin{verbatim}
A8 A7 A6 A5 A4 A3 A2 A1
B8 B7 B6 B5 B4 B3 B2 B1
C8 C7 C6 C5 C4 C3 C2 C1
D8 D7 D6 D5 D4 D3 D2 D1
E8 E7 E6 E5 E4 E3 E2 E1
F8 F7 F6 F5 F4 F3 F2 F1
G8 G7 G6 G5 G4 G3 G2 G1
H8 H7 H6 H5 H4 H3 H2 H1
\end{verbatim}
}

Le otto trasformazioni sono quindi definite dai metodi omonimi nelle due classi SquareSet e Board come
descritti nella tabella \ref{tab:bitboard_transformations}.
Inoltre le due classi hanno definiti i metodi \texttt{transformations} ed \texttt{anti\_transformations} che ritornano
un array di otto posizioni contenente le bitboard trasformate o anti-trasformate.

\begin{table}[h]
\centering
\caption{Trasformazioni della bitboard}
\label{tab:bitboard_transformations}
\begin{tabular}{|c|l|l|l|}
\hline
\textbf{Pos.} & \textbf{Descrizione} & \textbf{Trasformazione} & \textbf{Anti-Trasformazione} \\ \hline
0 & Rotazione zero gradi oraria  & \texttt{trans\_identity}               & \texttt{trans\_identity}               \\ \hline
1 & Rotazione 90 gradi oraria    & \texttt{trans\_rotate\_90c}            & \texttt{trans\_rotate\_90a}            \\ \hline
2 & Rotazione 180 gradi oraria   & \texttt{trans\_rotate\_180}            & \texttt{trans\_rotate\_180}            \\ \hline
3 & Rotazione 270 gradi oraria   & \texttt{trans\_rotate\_90a}            & \texttt{trans\_rotate\_90c}            \\ \hline
4 & Specchiatura asse verticale  & \texttt{trans\_reflection\_vertical}   & \texttt{trans\_reflection\_vertical}   \\ \hline
5 & Specchiatura diagonale H1-A8 & \texttt{trans\_reflection\_diag\_h1a8} & \texttt{trans\_reflection\_diag\_h1a8} \\ \hline
6 & Specchiatura asse orizontale & \texttt{trans\_reflection\_horizontal} & \texttt{trans\_reflection\_horizontal} \\ \hline
7 & Specchiatura diagonale A1-H8 & \texttt{trans\_reflection\_diag\_a1h8} & \texttt{trans\_reflection\_diag\_a1h8} \\ \hline
\end{tabular}
\end{table}

\subsection{Definizione e Composizione del Gruppo}
\label{sec:gruppo_diedrale}

Le trasformazioni geometriche che agiscono sulla scacchiera del Reversi formano un gruppo matematico
noto come \textbf{Gruppo Diedrale di ordine 8}, indicato con il simbolo $\bm{D_4}$.
Questo gruppo rappresenta l'insieme completo delle simmetrie di un quadrato.

\subsection{Composizione del Gruppo}
Il gruppo $G = \{T_0, T_1, \dots, T_7\}$ è composto da 8 elementi fondamentali:
\begin{itemize}
    \item \textbf{4 Rotazioni:} $0^\circ$ (l'elemento \textit{Identità} $e$), $90^\circ$, $180^\circ$ e $270^\circ$.
    \item \textbf{4 Riflessioni:} Orizzontale, Verticale e le due Diagonali.
\end{itemize}

\begin{table}[h]
\centering
\caption{Elementi del Gruppo Diedrale $D_4$ e relative Anti-trasformazioni.}
\label{tab:d4_inverses}
\begin{tabular}{@{}clllc@{}}
\hline
\textbf{Indice} & \textbf{Operatore} & \textbf{Descrizione Geometrica} & \textbf{Inverso ($T^{-1}$)} & \textbf{Ordine} \\ \hline
0 & $e$      & Identità ($0^\circ$)          & $T_0$ & 1 \\
1 & $r$      & Rotazione $90^\circ$ (CW)     & $T_3$ & 4 \\
2 & $r^2$    & Rotazione $180^\circ$         & $T_2$ & 2 \\
3 & $r^3$    & Rotazione $270^\circ$ (CW)    & $T_1$ & 4 \\
4 & $s$      & Riflessione Orizzontale       & $T_4$ & 2 \\
5 & $sr$     & Riflessione Diagonale (P)     & $T_5$ & 2 \\
6 & $sr^2$   & Riflessione Verticale         & $T_6$ & 2 \\
7 & $sr^3$   & Riflessione Diagonale (A)     & $T_7$ & 2 \\ \hline
\end{tabular}
\end{table}

L'ordine di un elemento \(T\in D_{4}\) rappresenta il ciclo di ripetizione della trasformazione geometrica.
Gli elementi di ordine 2, detti involuzioni, sono auto-inversi (\(T=T^{-1}\)), semplificando la logica di mappatura tra istanze simmetriche.

\subsection{Struttura Algebrica e Generatori}

Il gruppo diedrale $D_4$ può essere descritto in modo compatto attraverso i suoi \textbf{generatori}. Invece di considerare le otto trasformazioni come elementi isolati, le definiamo come combinazioni di due operazioni fondamentali.

\subsection{I Generatori: Rotazione e Riflessione}
Il gruppo $D_4$ è generato da due elementi fondamentali:
\begin{itemize}
    \item $r$: una rotazione di $90^\circ$ in senso orario (ordine 4, per cui $r^4 = e$).
    \item $s$: una riflessione rispetto all'asse orizzontale (ordine 2, per cui $s^2 = e$).
\end{itemize}

\subsection{Relazioni Fondamentali e Proprietà Non-Abeliana}
La struttura del gruppo è determinata dalla relazione di interazione tra questi due generatori. In $D_4$, vale la seguente relazione fondamentale:
\begin{equation}
    rs = sr^{-1} \quad (\text{ovvero } rs = sr^3)
\end{equation}
Questa uguaglianza dimostra formalmente che il gruppo \textbf{non è abeliano}: applicare una rotazione seguita da una riflessione ($rs$) non equivale ad applicare una riflessione seguita dalla stessa rotazione ($sr$). 

\subsection{Scomposizione degli Elementi}
Grazie ai generatori $r$ e $s$, l'insieme degli otto elementi di $D_4$ può essere ripartito in due sottoinsiemi disgiunti:
\begin{enumerate}
    \item \textbf{Sottogruppo delle Rotazioni (Ciclico):} 
    \begin{equation}
        R = \{e, r, r^2, r^3\}
    \end{equation}
    Questi elementi mantengono l'orientamento (chirale) originale della scacchiera.
    
    \item \textbf{Insieme delle Riflessioni (Classe Laterale):}
    \begin{equation}
        S = \{s, sr, sr^2, sr^3\}
    \end{equation}
    Ogni elemento di questo insieme rappresenta una riflessione. Algebricamente, una riflessione ``composta'' (come quella diagonale) è vista come l'atto di ribaltare la scacchiera ($s$) e successivamente ruotarla per raggiungere l'asse di simmetria desiderato.
\end{enumerate}

\subsection{Interpretazione Geometrica}
La notazione $sr^k$ spiega la natura delle simmetrie speculari:
\begin{itemize}
    \item $s$: Riflessione Orizzontale.
    \item $sr$: Riflessione rispetto alla Diagonale Principale.
    \item $sr^2$: Riflessione Verticale.
    \item $sr^3$: Riflessione rispetto alla Diagonale Anti-principale.
\end{itemize}
Questa classificazione è di fondamentale importanza per il software: permette di mappare ogni possibile simmetria della scacchiera utilizzando solo due funzioni atomiche (\textit{rotate} e \textit{reflection}) e le loro composizioni.

\subsection{Implementazione Computazionale (Parallel Bit Manipulation)}

Dal punto di vista dell'efficienza algoritmica, le trasformazioni del gruppo $D_4$ non sono implementate come rotazioni geometriche classiche, ma come sequenze di \textit{Parallel Bit Manipulation} (tecnica nota come \textit{Delta Swap}). Questa scelta permette di eseguire ogni simmetria in tempo costante $O(1)$ senza l'ausilio di cicli o tabelle di lookup pesanti.

\subsection{Composizione delle Trasformazioni}
Sebbene algebricamente il gruppo sia generato da rotazioni e riflessioni, nel software è più efficiente utilizzare le riflessioni assiali e diagonali come primitive atomiche. Le rotazioni vengono quindi derivate dalla composizione di queste ultime, come illustrato nella Tabella \ref{tab:code_mapping}.

\begin{table}[h]
\centering
\caption{Mappatura delle funzioni del software sugli elementi di $D_4$.}
\label{tab:code_mapping}
\begin{tabular}{lll}
\hline
\textbf{Elemento Algebrico} & \textbf{Funzione nel Software} & \textbf{Composizione Logica} \\ \hline
$e$ (Identità) & \texttt{trans\_identity} & - \\
$s$ (Riflessione Orizzontale) & \texttt{trans\_reflection\_horizontal} & Primitive \\
$sr^2$ (Riflessione Verticale) & \texttt{trans\_reflection\_vertical} & Primitive \\
$sr$ (Riflessione Diagonale) & \texttt{trans\_reflection\_diag\_h1a8} & Primitive \\
$r$ (Rotazione $90^\circ$ CW) & \texttt{trans\_rotate\_90c} & \texttt{refl\_diag\_h1a8} $\circ$ \texttt{refl\_horiz} \\
$r^2$ (Rotazione $180^\circ$) & \texttt{trans\_rotate\_180} & \texttt{refl\_horiz} $\circ$ \texttt{refl\_vert} \\
$r^3$ (Rotazione $90^\circ$ ACW) & \texttt{trans\_rotate\_90a} & \texttt{refl\_horiz} $\circ$ \texttt{refl\_diag\_h1a8} \\ \hline
\end{tabular}
\end{table}

\subsection{Analisi dell'Efficienza}
L'approccio utilizzato riduce il \textit{branching} della CPU. Operazioni come \texttt{trans\_reflection\_diag\_h1a8} utilizzano maschere costanti (\textit{k1, k2, k4}) e operazioni di \textit{shift} e \textit{XOR} per scambiare interi blocchi di bit simultaneamente. Questo garantisce che la valutazione dei pattern rimanga performante anche durante ricerche profonde nell'albero di gioco.

